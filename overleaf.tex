\begin{abstract}
    L\'{a}szl\'{o} Fejes T\'{o}th and Alad\'{a}r Heppes proposed the following generalization of the kissing number problem. Given a ball in $\mathbb{R}^d$, consider a family of balls touching it, and another family of balls touching the first family. Find the maximal possible number of balls in this arrangement, provided that no two balls intersect by interiors, and all balls are congruent. They showed that the answer for disks on the plane is $19$. They also conjectured that if there are three families of disks instead of two, the answer is $37$. In this paper we confirm this conjecture.
\end{abstract}

\section{Introduction}

A collection $\mathcal C$ of convex bodies in $\mathbb R^d$ is called a \emph{packing} if no two of them have an interior point in common. We assume that two convex bodies of $\mathcal{C}$ may touch each other. % We will also only consider packings of unit disks~--- that is, disks of unit diameter.
In the current paper, we are only interested in finite packings of disks (and balls) of unit diameters, and therefore, we omit the phrase ``of unit diameter'' most of the time.

% If not constraining ourselves to two dimensions, different types of packings are interesting to people.
% One of the classical problems of discrete geometry is the so-called Kepler conjecture asserting that the maximal possible density of a packing of unit balls in $\mathbb{R}^3$ is \frac{\pi}{3\sqrt{2}}. Since then, various other problems \cite{thue1892om, thue1911dichteste} about optimal packings were considered. For example, L\'{a}szl\'{o} Fejes T\'{o}th has shown \cite{Fejes_Toth2013-wc} that the density of an optimal packing of the space can be reduced to a finite case analysis.

% In 1911, Thue~\cite{thue1892om, thue1911dichteste} showed the upper bound on the packing density of disks of the same size in the plane. In 1950, this result was generalized by Fejes Toth~\cite{toth1950some} for translates of a convex body. These works had a great influence on the research of packing problems. In the current paper, we are interested in packings of unit-diameter disks.

% When considering centers of balls in finite packings, some special properties and problems may arise. For example, for a finite set of points one can define a \emph{minimum-distance graph}, where vertices are the points of the set, and edges are drawn between all pairs of points with the minimum distance among all pairwise distances in the set. It it known \cite{min-distance-edges} that the maximal possible number of edges in a minimum-distance graph of a set of $n$ points of the plane equals $\lfloor3n - \sqrt{12n - 3}\rfloor$. If there is at least one pair of touching balls in the packing, the minimum-distance graph of their centers becomes the \emph{unit-distance graph}, defined similarly. \cite{Swanepoel2018} may serve as a survey on these concepts.

The concept of \emph{minimum-distance graphs} is tightly connected with packings of balls.
% In the study of disc packing problems, it is useful to consider the so-called \emph{minimum distance graph}.
Given a finite set of points in $\mathbb{R}^d$, the minimum-distance graph is the graph whose vertices are points of the set and edges are drawn between all pairs of points within the minimum distance among all pairwise distances in the set. In particular, if a packing of congruent balls in $\mathbb{R}^d$ contains at least one pair of touching balls, then there is a correspondence between pairs of touching balls and edges in the minimum-distance graph induced by the centeres of the balls. % Harborth~\cite{min-distance-edges} showed that the number of edges in a minimum-distance graph in the plane on $n$ vertices does not exceed $\lfloor3n - \sqrt{12n - 3}\rfloor$, and moreover, this upper bound is tight.
We refer the interested reader to the recent survey~\cite{Swanepoel2018} on various distance problems in discrete geometry, including those on packings.

% If $X$ is a normed space, then the maximum possible degree of a vertex of a minimum-distance graph in $X$ is called the \emph{kissing number of $X$}. In other words, the kissing number of a space shows how many unit balls can simultaneously touch (or kiss) another unit ball and not overlap with each other. Finding the kissing number of $\mathbb{R}^n$ is yet another classical problem, which originates from 1694, when Isaac Newton and David Gregory had a disagreement about that of the three-dimensional space. The kissing number of $\mathbb{R}^2$ is obviously $6$. Now it is known \cite{bender1874bestimmung, gunther1875stereometrisches, schutte1952problem, leech1956problem} that the kissing number of $\mathbb{R}^3$ is $12$. Coxeter \cite{coxeter1963upper} showed some bounds on kissing numbers of $\mathbb{R}^n$ for some other $n$.

Another concept closely related to the local structure of a packing of balls and minimum-distance graphs is the \emph{kissing number}. Recall that the kissing number in $\mathbb{R}^d$ is the maximum number of $d$-dimensional congruent balls with non-intersecting interiors that can touch a given ball of the same size. One can easily see that the kissing number of a space is the maximum possible degree of a vertex in a minimum-distance graph of any set of points from that space.
% When studying packages of balls, it may be necessary to understand the local structure. For example, it is important to find the \emph{kissing number}, that is, the maximum number of unit balls that can touch a given unit ball.
It is clear that the kissing number in the plane is $6$. One of the classical questions in mathematics is the thirteen spheres problem raised by Isaac Newton and David Gregory whether the kissing number in $\mathbb{R}^3$ is $12$ or $13$, which was first settled by K.~Sch\"{u}tte and B.\,L.~Van~der~Waerden in~\cite{schutte1952problem}. % We refer the interested reader to the survey~\cite{coxeter1963upper} on the kissing number.

L. Fejes T\'{o}th and A. Heppes~\cite{toth_heppes} considered a generalization of the kissing number problem. Given a ball and a family of balls of the same size, all touching the initial one, add another family of balls of the same size that all of them touch at least one previous ball. No two balls may intersect by interiors. Find the maximal number $T_d$ of balls, except the first one, in such an arrangement. Fejes T\'{o}th and Heppes proved some lower and upper bounds for $T_2$, $T_3$, and $T_4$; for instance, $T_2 = 18$. In the last paragraph of the introduction of \cite{toth_heppes}, they also wrote:
\begin{quote}
We can continue the process of
successively enlarging a bunch of balls, but in the next step the problem becomes extremely intricate, even in the plane.
\end{quote}
% We generalize this concept in another direction.
Z.~F\"{u}redi and P.\,A.~Loeb~\cite[last paragraph of Section~6]{furedi} attribute the following conjecture to L.\,Fejes~T\'{o}th and A.~Heppes: The similar three-layer configuration in the plane contains at most $36$ disks (except the first given disk). The goal of this paper is to confirm this conjecture.

Let us introduce this problem formally. Given a packing $\mathcal{P}$ of disks, for two disks $A$, $B\in\mathcal{P}$ we define the \emph{kissing distance} between $A$ and $B$, as the smallest $n$ such that there is a sequence of disks $D_0$, $D_1$, \ldots, $D_n\in\mathcal{P}$, where $A = D_0$, $B = D_n$, and for $1\leq i\leq n$ the disks $D_{i-1}$ and $D_i$ touch each other. If there is no such sequence, then we set the kissing distance between $A$ and $B$ to $\infty$. We say that a packing $\mathcal{P}$ \emph{has kissing radius $n$} if there is a disk $D\in\mathcal{P}$, which we call \emph{the source} or \emph{the 0-disk}, such that for every other disk $B$ the kissing distance between $D$ and $B$ does not exceed $n$. Denote by $f(n)$ the maximum number of disks in a packing of kissing radius $n$.

% \textbf{Todo:} add some back story. Most of the time we consider disks of unit diameter, so unless specified, all disks are assumed to have unit diameter. An arrangement of disks is called \textit{packing} if no two of them overlap. For a packing $\mathcal{P}$ and $A$, $B\in\mathcal{P}$, the \textit{kissing distance} between $A$ and $B$, denoted by $\dist_{\mathcal{P}}(A, B)$, is the smallest $n$ such that there is a sequence of disks $C_0$, $C_1$, \ldots, $C_n\in\mathcal{P}$, where $A = C_0$, $B = C_n$, such that for $1\leq i\leq n$, the disks $C_{i-1}$ and $C_i$ touch each other; if there is no such sequence, then we set $\dist_{\mathcal{P}}(A, B) = \infty$. We say that a packing $\mathcal{P}$ \textit{has kissing radius $n$} if there is a disk $C_0\in\mathcal{P}$, which we call \textit{the source} or \textit{the 0-disk}, such that for every other disk $B$ we have $\dist_{\mathcal{P}}(C_0, B)\leq n$. Denote by $f(n)$ the maximum number of disks in a packing of kissing radius $n$.

In terms of minimum-distance graphs, $f(n)$ is the maximum possible number of vertices in the induced subgraph with just the vertices at distance $n$ or less from a given vertex. % For instance, $f(1)$ is one plus the kissing number of the plane.

The hexagonal arrangement of disks gives the trivial lower bound $f(n) \geq 1 + {3n(n+1)}.$ Since the kissing number of the plane is $6$, we have $f(1) = 7$. According to~\cite{toth_heppes}, we have $f(2) = 19$. The main result of this paper is to confirm the conjecture $f(3) = 37$ mentioned in~\cite{furedi}.
% Moreover, they conjectured that $f(3) = 1 + 3\cdot3\cdot(3+1) = 37$; see~\cite[last paragraph of Section~6]{furedi}.
% We confirm this conjecture.

\begin{theorem}\label{theorem:f_of_3} $f(3) = 37$.
\end{theorem}

% Let a packing $\mathcal{P}$ have a kissing radius $n$. Therefore, the disk $D$ of radius $n + 1/2$ concentric with the source disk of $\mathcal{P}$ covers all disks of $\mathcal{P}$. Let $\mathcal{P}'_n$ be any densest packing in $D$ \textit{without} any restriction on its kissing radius. We prove that $f(n)$ is asymptotically close to the size of $\mathcal{P}'_n$.

% % We prove that for large $n$ the function $f(n)$ is asymptotically close to the densest packing of $D$.

% \begin{theorem}\label{theorem:f_asymptotic}
% $f(n) = (1 - o(1))\dfrac{2\pi}{\sqrt{3}}n^2$.
% \end{theorem}

It is also worth mentioning that this pattern does not generalize on all $n$. In fact, $f(n) = \dfrac{2\pi}{\sqrt{3}}n^2\cdot(1 + o(1))$, but the paper with this result is in progress at the moment of writing.

This paper is organized as follows. In Section~2 we introduce the required notation and two key lemmas, Lemma~\ref{lemma:master} and Lemma~\ref{lemma:good_curve}. After this we deduce Theorem~\ref{theorem:f_of_3} from these lemmas. In Section~3 we prove Lemma~\ref{lemma:master}. The proof of Lemma~\ref{lemma:good_curve} is separated into constructing a curve of certain type (Section~4), reducing Lemma~\ref{lemma:good_curve} to a certain local inequality Claim~\ref{lemma:can-construct-without-alpha} (Section~5), and, finally, establishing this inequality (Section~6).

% \textbf{Todo: mention lemma 2.1 and compare it with the triangle inequality, also add the detailed plan of section 2.} The paper is organized as follows. In Section 2 (todo: make this number relative) we prove Theorem \ref{theorem:f_of_3}, in Section 3 we give an example of an arrangement showing $f(n) > 1 + 3n(n+1)$, and in Section 4 we adapt this example to prove Theorem \ref{theorem:f_asymptotic}. 

\medskip \textbf{Acknowledgements.}
The author thanks Alexey Balitskiy, Dmitry Krachun and Alexandr Polyanskii for helpful discussions and comments on drafts of this paper. The author also thanks Alexandr Polyanskii for simplifying the proof of Case~\ref{subsec:case-000}.

\section{Proof of Theorem \ref{theorem:f_of_3} modulo the key lemmas}

% \subsection{Preliminaries}

% \subsubsection{Definitions}

\subsection{Definitions}

Let $\mathcal{P}$ be a packing of kissing radius $n$ with $D$ being its source. A disk $D' \in \mathcal{P}$ is called a \textit{$d$-disk} if the kissing distance between it and $D$ equals $d$. Note that this definition justifies the name ``0-disk''. We call the set of $d$-disks the \textit{$d\textsuperscript{th}$ layer}. % ; with a slight abuse of notation, we sometimes use the same word \textit{layer} to refer to the corresponding set of centers intstead of the disks themselves. %We may also abuse the definition of the $d$-th layer to express the corresponding set of centers instead of the disks theirselves.

Consider any $k$-disk $A$, where $k > 0$. There is at least one disk $B$ from the $(k-1)$\textsuperscript{th} layer such that $A$ and $B$ touch each other. Pick any of these disks and call it the \textit{parent} of the disk $A$.
Denote the parent of the disk $A$ by $\parent(A)$.
In particular, $C_0$ is the parent of every $1$-disk. We will also say that each disk is a \emph{child} of its parent. A disk without children is called \emph{childfree}.
% We say that a disk $A$ is a \textit{parent} of a disk $B$ if $\dist(C_0, A) + 1 = \dist(C_0, B)$, and disks $A$ and $B$ touch each other. In particular, $C_0$ is a parent of every $1$-disk. A disk can have multiple parents, but in this case we say that it only has one of them and pick it arbitrarily.

The crucial role in the argument will be played by curves of special type, which we now introduce. A curve $\gamma = \{\gamma(t)\,\colon\,t\in[0, \ell]\} \subset \mathbb{R}^2$ is \textit{sparse-centered} if the following conditions hold.

\begin{enumerate}[label=(\alph*)]
  \item There exists a sequence of numbers $t_0 = 0 < t_1 < \ldots < t_m=\ell$ such that $\gamma([t_{i-1}, t_i])$ is a circle arc of unit radius with center $c_i$ for all $i\in\{1, \ldots, m\}$;
  \item $\gamma(t)$ is a natural parametrization of $\gamma$, that is, $|\dot{\gamma}(t)| = 1$ for $t\in[0, \ell]\setminus\{t_i\}_{i=0}^m$, and thus, $|\gamma| = \ell$;
  \item For all $i\in\{1, \ldots, m\}$ and $t\in(t_{i-1}, t_i)$, the cross product $(\gamma(t) - c_i)\times \dot{\gamma}(t)$ is positive, that is, each arc is directed counterclockwise;
  \item For all $i$, $j\in\{1, \ldots, m\}$, either $c_i = c_j$ or $|c_j - c_i|\geq 1$.
\end{enumerate}

In the sequel we often omit the word ``sparse-centered'' when talking about curves.

\subsection{Outline of the proof of Theorem \ref{theorem:f_of_3}}

Consider an arbitrary packing of unit diameter disks of kissing radius $3$. Remove all $3$-disks and childfree $1$-disks from it and denote by $\mathcal{P}$ the remaining packing. We may assume that $\mathcal{P}$ has at least two $2$-disks, as in the opposite case a very rough upper bound on the number of disks in the initial packing would be $f(2) + 6 = 25$, where $6$ is the maximal number of $3$-disks, all touching the $2$-disk. Denote by $C_i$ the number of $i$-disks of $\mathcal{P}$.

Then, consider the union $S$ of open disks of unit radii whose centers are the centers of disks from $\mathcal{P}$ (recall that the disks of the initial packing are of unit diameter, not of unit radius). One can see that centers of the disks we excluded lie on the boundary of $S$.

% Then we consider the union $S$ of open disks of unit radii whose centers are the centers of $0$-, $1$-, and $2$-disks (note that the disks of $\mathcal{P}$ were of unit diameter, not of unit radius).

% We use this notation throughout the section.

% We construct a closed, possibly self-intersecting, sparse-centered curve $\gamma$ containing the whole boundary of $S$. Then we prove that the centers of the excluded disks break $\gamma$ into parts of length at least $\pi/3$. Finally, we establish the following inequality.

% $$C_1 + C_2 + \frac{|\gamma|}{\pi/3} \leq 36.$$

% In some simple cases $\gamma$ coincides with $\partial S$, the boundary of $S$; in general (for instance, when $S$ is not simply connected), $\gamma$ covers but does not coincide with $\partial S$.

Theorem~\ref{theorem:f_of_3} follows from the lemmas below.

\begin{lemma}
Let $\gamma$ be a sparse-centered curve with endpoints $a$ and $b$. If $|b - a|\geq 1$ then the length of $\gamma$ is at least $\pi/3$.
\label{lemma:master}
\end{lemma}

\begin{lemma}
\label{lemma:good_curve}
There exists a closed sparse-centered curve $\gamma$ covering the boundary of $S$ and satisfying the inequality
$$C_1 + C_2 + \frac{|\gamma|}{\pi/3} \leq 36,$$
where $|\gamma|$ stands for the length of $\gamma$.
\end{lemma}
% \begin{lemma}
% If the inequalities above hold then the curve we construct satisfies the condition of Lemma \ref{lemma:good_curve}.
% \end{lemma}

\begin{proof}[Deriving Theorem~\ref{theorem:f_of_3} from Lemma~\ref{lemma:master} and Lemma~\ref{lemma:good_curve}]
Consider the curve $\gamma$ from Lemma~\ref{lemma:good_curve}. All centers of excluded disks of the initial packing belong to $\gamma$. Moreover, by Lemma~\ref{lemma:master}, they split $\gamma$ into parts, each of length at least $\pi/3$. Therefore, there is at most $\frac{|\gamma|}{\pi/3}$ excluded disks, which together with Lemma~\ref{lemma:good_curve} implies that the number of disks in the initial packing does not exceed

$$1 + C_1 + C_2 + \frac{|\gamma|}{\pi/3} \leq 37.$$
\end{proof}

\section{Proof of Lemma \ref{lemma:master}}

\begin{observation}
Given points $c_1$, \ldots, $c_n$, consider all sparse-centered curves with centers at a subset of $\{c_i\}$ and with endpoints at least $1$ apart.
% Then if there is at least one such curve, then there are such curves of shortest possible length.
There exist such curves of shortest possible length.
\end{observation}

\begin{proof}
%First of all, we notice that it suffices to prove the statement only for simple (that is, non-self-intersecting) curves. Indeed, if there is a valid curve, then there exists a shorter self-intersecting simple valid curve.
%
Let $I$ be the set of the intersection points between all circles of unit radius centered at $c_1$, \ldots, $c_n$. Some of these circles are partitioned into closed arcs by the elements of $I$ (each arc is bounded by two adjacent points from $I$ on its circle). Let $A$ be the set of all these arcs, directed counterclockwise. The set $A$ is obviously finite.

Since any arc of length $\pi/3$ is a valid curve, without loss of generality, we may assume that valid curves have length at most $\pi/3$ and endpoints at least $1$ from each other. Since the set of all arcs $A$ is finite, we may derive that every such curve consists of at most $m$ arcs for some positive $m$.
% Indeed, each inner arc is at least $\min\limits_{a\in A}|a|$ long, and there are at most two utmost arcs.
Using the standard compactness argument, we obtain that there is a shortest valid curve consisting of at most $m$ circular arcs.
\end{proof}

Let $\gamma$ be the sparse-centered curve from Lemma~\ref{lemma:master}, and $t_0$, \ldots, $t_m$ be as in the definition of sparse-centered curves.
% If it is not the shortest possible sparse-centered curve with the same centers, replace it with the shortest one.
Without loss of generality, we may assume that $\gamma$ is a shortest sparse-centered curve satisfying the conditions of Lemma~\ref{lemma:master}.

\begin{observation}
There do not exist $0 < x_1 < x_2 < \ell$ such that the derivatives $\dot{\gamma}(x_1)$ and $\dot{\gamma}(x_2)$ exist and are equal.
\end{observation}

\begin{proof}
Suppose the contrary. Let $p_1$ and $p_2$ be the points $\gamma(x_1)$ and $\gamma(x_2)$, respectively. Let $c_1$ and $c_2$ be the centers of the arcs containing $p_1$ and $p_2$. According to the definition, $\dot{\gamma}(x_1)$ equals $p_1 - c_1$ rotated by the angle $\pi/2$ counterclockwise, and the same holds for $\dot{\gamma}(x_2)$ and $p_2 - c_2$. Since $\dot{\gamma}(x_1) = \dot{\gamma}(x_2)$, we have $p_1 - c_1 = p_2 - c_2$ and, therefore, $c_2 - c_1 = p_2 - p_1$. There are two cases.

\begin{itemize}
    \item Suppose that $c_1 = c_2$. Then the curve $\gamma([0, x_1)\cup[x_2, \ell])$ is a shorter sparse-centered curve with endpoints being $1$ apart. This contradicts the minimality of $\gamma$.
    
    \item Otherwise, $|c_2 - c_1| \geq 1$, and hence, $|p_2 - p_1|\geq 1$. But in this case $\gamma([x_1, x_2])$ is, again, a sparse-centered curve with length strictly less than $\ell$, which contradicts the minimality of $\gamma$ just as in the first case.
\end{itemize}

In both cases we arrive at a contradiction, which finishes the proof of the observation.
\end{proof}

Denote the set $\left\{\dot{\gamma}(t)\,\colon\,t\in(0, \ell)\setminus\{t_1, \ldots, t_m\}\right\}$ of all directions along $\gamma$ by $S$.
We can introduce a measure $\mu$ on the Borel subsets of the unit circle, assigning $\mu(s)$ to be the length of $\{\gamma(t)\,\colon\,\dot{\gamma}(t)\in s\}$.

\begin{corollary}
$\mu(S) = \ell$.
\end{corollary}

Let us note that $$b - a = \int\limits_0^{\ell}\dot{\gamma}(t)\,\mathrm{d}t = \int\limits_Sv\,\mathrm{d}\mu(v).$$

Let $\proj_{ab}(v)$ be the projection of vector $v$ on the line through $a$ and $b$. Since $\gamma(t)$ is parametrized naturally, $|b - a|$ does not exceed

$$\left|\int\limits_Sv\,\mathrm{d}\mu(v)\right| = \left|\int\limits_S\proj_{ab}(v)\,\mathrm{d}\mu(v)\right| \leq \int\limits_{-\ell/2}^{\ell/2}\cos(\varphi)\,\mathrm{d}\varphi = 2\sin(\ell/2).$$

\medskip Since $|b - a| = 1$, we have $\sin(\ell/2)\geq 1/2$ or $\ell\geq\pi/3$, which finishes the proof of Lemma~\ref{lemma:master}.

\textit{Remark.} The requirement of the curve going counterclockwise is crucial. For example, otherwise the curve can be composed of two arcs of length $2\arcsin(1/4)$ each, one going clockwise and the other smoothly continuing the first in the counterclockwise direction. The centers of the corresponding disks are ``at different sides'' of the curve, and, quite clearly, are sufficiently far from each other; see Figure~\ref{fig:incorrect-curve}.

\begin{figure}[h!]
    \centering
    \includegraphics{pics/incorrect-curve.mps}
    \captionsetup{width=.7\textwidth}
    \caption{An example of an invalid curve}
    \label{fig:incorrect-curve}
\end{figure}

\section{Building a curve in Lemma \ref{lemma:good_curve}}

Our current goal is to construct a sparse-centered curve containing the boundary $\partial S$. In Sections~5 and~6 we show that it is sufficiently short.
In the simplest case, we could just use $\partial S$ as such a curve, but in case if $S$ is not simply connected, we cannot do this; see Figure~\ref{fig:partial-s}.

The packing $\mathcal{P}$ of kissing radius $2$ corresponds to a directed plane tree (a planar drawing of a tree) as follows; see Figure~\ref{fig:tree}. The vertices of the tree are the centers of the disks of $\mathcal{P}$. For each disk, except the source, its vertex is connected to the vertex of its parent by an edge, we consider all edges as unit vectors directed from the parent towards its child. The resulting graph is a tree as there is a path from any disk to the source and it has no undirected cycles; otherwise the greatest-layer disk of this cycle has two parents. Since $\mathcal{P}$ is a packing, all edges have unit length and the distance between any two vertices is at least $1$. This implies that no two edges intersect, except, possibly, at their common endpoint. We call this graph the \textit{$\mathcal{P}$-tree}. It can be seen that the $\mathcal{P}$-tree is a subgraph of the minimum-distance graph of the centers of the disks.

\begin{figure}[h!]
    \centering
    \begin{subfigure}[t]{.48\textwidth}
    \includegraphics[width=.95\textwidth]{pics/tree.mps}
    \caption{The $\mathcal{P}$-tree}
    \label{fig:tree}
    \end{subfigure}
    \begin{subfigure}[t]{.48\textwidth}
    \includegraphics[width=.95\textwidth]{pics/traversal.mps}
    \caption{A traversal example. Some disks are counted multiple times, one time per visiting the corresponding vertex}
    \label{fig:traversal}
    \end{subfigure}
    \caption{}
\end{figure}

Let us fix a traversal of the $\mathcal{P}$-tree; see Figure~\ref{fig:traversal}.
Since it is a plane tree, we can consider it as a planar graph with the only face. Traversing the boundary of this face, we get a cyclic ordering $(c_1, c_2, \ldots, c_n)$ of the set of vertices with multiplicities. We assume that all indices are taken modulo $n$. The number of times a vertex occurs in the ordering is its degree\footnote{this ordering is basically a depth-first search (DFS) traversal of the configuration tree.}.
This implies that every $2$-disk occurs exactly once in this order.
Let $(D_1, \ldots, D_n)$ be the corresponding ordering of the disks from $\mathcal{P}$. % ; we also assume $D_{n+1} = D_1$, $D_{n+2} = D_2$, etc; the same about $c_i$ for $i > n$.
Also, let $B_1$, \ldots, $B_n$ be the open disks of unit radius centered respectively at $c_1$, \ldots, $c_n$. %Denote the center of the disk $D_i$ by $c_i$. 
Recall that $\bigcup_{i=1}^nB_i = S$.

Given $i,j\in[n]$, call a \textit{subsegment} $\mathcal{D}_{ij}$ (of our traversal) the sequence $(D_i, \ldots, D_j)$, if $i \leq j$, or $(D_i, \ldots, D_n, D_1, \ldots, D_j)$, otherwise. In this paper, we only consider subsegments where $D_i$ and $D_j$ are the only $2$-disks in $\mathcal{D}_{ij}$. For simplicity, we assume that $i\leq j$ in the rest of the paper.

It can be seen that $\mathcal{D}_{ij}$ consists of either $3$ or $5$ disks. Indeed, if $\parent(D_i) = \parent(D_j)$ (or, equivalently, the source $D$ is not in $\mathcal{D}_{ij}$), then $\mathcal{D}_{ij} = (D_i, \parent(D_i), D_j)$. Otherwise, $$\mathcal{D}_{ij} = \left(D_i, \parent(D_i), D, \parent(D_j), D_j\right).$$

We assume that the $0$-disk $D$ is centered at the origin $c$. By analogy with $D$, denote by $B$ the disk of unit radius centered at $c$. For any $i$ such that $D_i$ is a $2$-disk, denote by $f_i$ the farthest point of the disk $B_i$ from the origin $c$.
% Since $c$ is not the center of $B_i$, the point $f_i$ is well-defined.
Formally,

$$f_i = c_i\cdot\left(1 + \frac{1}{|c_i|}\right).$$

By $[ab)$ we denote the ray starting at point $a$ and passing through $b$. If $D_i$ and $D_j$ are two consecutive occurrences of $2$-disks, consider segments $c_ic_{i+1}$, \ldots, $c_{j-1}c_j$, and two rays $[c_if_i)$ and $[c_jf_j)$. Their union divide the plane into two parts; denote by $R_{ij}$ the closed one for which $(c_i, \ldots, c_j)$ is the clockwise ordering of vertices.
% Since $\mathcal{P}$ was obtained from the initial packing by removing $3$-disks and childfree $1$-disks, currently each region contains either $3$ or $5$ vertices.
As we already know, each region contains either $3$ or $5$ vertices.

The plane is then partitioned into such regions (see Figure~\ref{fig:division-into-regions}), and for each of them we will construct a sparse-centered curve $\gamma_{ij}\subset R_{ij}$ with centers among $\{c_i, \ldots, c_j\}$, having endpoints $f_i$ and $f_j$ and containing $\partial S\cap R_{ij}$. After this, we consider $\gamma$ as the union of all $\gamma_{ij}$.

\begin{figure}[h!]
    \centering
    \begin{subfigure}{.4\textwidth}
    \includegraphics[width=\textwidth]{pics/fig.mps}
    \caption{$\partial{S}$ is in bold}
    \label{fig:partial-s}
    \end{subfigure}
    \begin{subfigure}{.4\textwidth}
    \includegraphics[width=\textwidth]{pics/fig_reg.mps}
    \caption{Division into regions}
    \label{fig:division-into-regions}
    \end{subfigure}
    % \caption{A disk arrangement and the associated partition}
    \caption{}
    \label{fig:regions}
\end{figure}

% \subsubsection{Auxiliary lemmas}

% comment: moved up
% To construct $\gamma$ by parts in this way, we denote by $f_i$ the farthest point of the 2-disk $D_i$ from $c$. Since $c$ is not the center of $D_i$, the point $f_i$ is well-defined. Formally, $$f_i = c_i + (c_i - c)\cdot\left(1 + \frac{1}{|c_i - c|}\right).$$

First, we prove the following auxiliary lemmas.

\begin{observation}
If $D_i$ is a $2$-disk then no point of the ray $[c_if_i)$ outside the segment $c_if_i$ belongs to any disk. In particular, $f_i\in\partial S$.
% Moreover, $f_i$ is the only point where the ray $[c_if_i)$ intersects $\partial S$.
\label{lemma:far}
\end{observation}

\begin{proof}
Let $g_i$ be a point of the ray $[c_if_i)$ outside the segment $c_if_i$. Suppose that $g_i$ lies in $S$.
Assume that $g_i$ is covered by some disk $D'_k$, and let $D = D'_0$, \ldots, $D'_k$ ($1\leq k\leq 2$) be the disks corresponding to the path in the $\mathcal{P}$-tree from the origin to the center of $D'_k$; denote by $c'_j$ the center of $D'_j$.
% Consider the sequence $(c'_0, \ldots, c'_k, g_i)$. Since $k\leq 2$, this sequence consists of at most four points.
Since for any $j\in[k - 1]$, all sides of the triangle $c'_jc'_{j+1}c_i$ have lengths at least $1$ and $|c'_{j+1} - c'_j| = 1$, the angle between the vectors $c'_j - c_i$ and $c'_{j+1} - c_i$ is at most $\pi/3$. Similarly, the angle between $c'_k - c_i$ and $g_i - c_i$ is strictly less than $\pi/3$. Since $k\leq 2$, these inequalities contradict the fact that the angle between $c'_0 - c_i$ and $g_i - c_i$ equals $\pi$. Thus, $g_i\notin S$.

Since all points of the segment $c_if_i$ but $f_i$ clearly lie in $B_i\subset S$, we have $f_i\in\partial{S}$.
\end{proof}

% \textbf{Remark.} The same proof shows that any point on the ray $[c_if_i)$ behind $f_i$ cannot be covered by any closed disk of the arrangement.

\begin{observation}
Let $\mathcal{D}_{ij}$ be a subsegment. Let $S_{ij}$ be the union of $B_i$, \ldots, $B_j$. Then $S\cap R_{ij} = S_{ij}\cap R_{ij}$; see Figure~\ref{fig:only_inner}.

% Let $D_i$ and $D_j$ be two consecutive 2-disks in the cyclic ordering $(D_1, \ldots, D_n)$. Consider the part $R$ of the plane which is bounded by segments $c_ic_{i+1}$, $c_{i+1}c_{i+2}$, \ldots, $c_{j-1}c_j$, and two rays $[c_if_i)$ and $[c_jf_j)$, and which does not contain the rest of the tree (see Figure~\ref{fig:only_inner}). Let $S'$ be the union of disks of unit radius with centers at $c_i$, $c_{i+1}$, \ldots, $c_j$. Then $S\cap R = S'\cap R$.
\label{lemma:only_inner}
\end{observation}

\begin{figure}[h!]
    \centering
    \includegraphics{pics/only_inner.mps}
    \captionsetup{width=.7\textwidth}
    \caption{The part of the boundary of the dashed disk in the region is completely in the gray area, and therefore, add nothing to the border}
    \label{fig:only_inner}
\end{figure}

% An informal reformulation of this lemma is that each time we consider a particular subsegment, its disks give us all information about $S$ that we need to build a corresponding part of $\gamma$.

\begin{proof}
Suppose that the set $(S\setminus S_{ij})\cap R_{ij}$ is not empty and $p$ is any point of this set.
Let $c'$ be the center of any disk containing the point $p$. Since $c'\not\in R_{ij}$ and $p\in R_{ij}$, the segment $c'p$ intersects $\partial{R_{ij}}$. Consider two cases.

If $c'p$ intersects some segment $c_kc_{k+1}$ then by the triangle inequality one of the segments $c_kp$ and $c_{k+1}c'$ is strictly shorter than $1$, which cannot be the case.

Otherwise $c'p$ intersects one of the rays $[c_if_i)$ and $[c_jf_j)$. Without loss of generality assume that it is $[c_if_i)$. By Lemma \ref{lemma:far} one can assume that the intersection is on the segment $c_if_i$. Analogously, at least one of the segments $c_ip$ and $f_ic'$ is strictly shorter than $1$, which also leads to a contradiction.
\end{proof}

%\subsubsection{Notation}

Denote $\partial S\cap R_{ij}$ by $\sigma_{ij}$.
We say that a disk $B_k$ is \textit{involved in $\sigma_{ij}$} if $\partial{B_k}\cap \sigma_{ij}$ is nonempty; that is, if there is a part of its boundary in $R_{ij}$ that is not covered by interiors of other disks. % We also say that a disk $B$ is just \textit{involved in the boundary} if $\partial{B}\cap S_{ij}$ for some $S_{ij}$ is nonempty and has nonzero length. % there is a part of its boundary (possibly of zero length) in the current region which is not covered by interiors of other disks.
Note that $B_i$ and $B_j$ are always involved in $\sigma_{ij}$.
This follows from the fact that $f_i$ and $f_j$ are always points of the boundary.

% Finally, we are ready to show how to build $\gamma_{ij}$ having centers among $\{c_i, \ldots, c_j\}$ and endpoints $f_i$ and $f_j$.
% We will separate the cases whether the source does or does not occur in $\mathcal{D}_{ij}$.
% Recall that in order to construct $\gamma$ we consider regions for different subsegments and define the part of $\gamma$ belonging to the region.
% How we build each separate part depends heavily on the structure of the corresponding subsegment. In particular, there can be zero or one occurrence of the source in the subsegment.

% Note that there cannot be more than one occurrence of the source in the subsegment, because we excluded all childfree $1$-disks and none of the $2$-disks.
% Next, there is a major difference between the cases when some angles are greater than or no more than $\pi/3$ or $2\pi/3$.
% To construct the part of the curve we introduce the following angular notation.

% Note that we can assume that in all regions $k\leq 1$. Indeed, suppose that there exists at least one $1$-disk without children. For each such disk $D_i$ we create a unique $2$-disk, centered at $f_i$. According to Lemma~\ref{lemma:far}, if we add all new disks step by step, then the packing remains valid after each step. After this operation, each region with $k > 1$ occurrences of the source in the corresponding subsegment is separated into $k$ regions, each having exactly one occurrence of the source. However, the total upper bound of their $\gamma_{ij}$ is $3\psi - \frac{2\pi}{3}k + 2\pi$, which is the same as it was before separating. Thus, to bound the total number of disks, it suffices to consider packings with no region having $k > 1$.

% \begin{lemma}\label{lemma:construct-by-involved}
% If $i = k_1 < k_2 < \ldots < k_l = j$ is the sequence of indices of all involved disks and some of non-strictly involved disks (possibly none, possibly all of them), then we can construct a required curve with direction jumps between $B_{k_1}$ and $B_{k_2}$, between $B_{k_2}$ and $B_{k_3}$, etc.
% \end{lemma}

\begin{claim}\label{lemma:can-construct-gamma-ij}
Let $\mathcal{D}_{ij}$ be a subsegment. Let $i = k_1 < \ldots < k_l = j$ be a sequence of indices of all disks from $\mathcal{D}_{ij}$ that are involved in $\sigma_{ij}$. Let $\gamma_{ij}$ be the curve constructed in the following way:
\begin{enumerate}
    \item $\gamma_{ij}$ starts at $f_i$ and goes counterclockwise along $\partial B_i$ until meeting $\partial B_{k_2}\cap R_{ij}$,
    \item Then $\gamma_{ij}$ goes counterclockwise along $\partial B_{k_2}$ until meeting $\partial B_{k_3}$,
    \item Then it goes in the following manner along $B_{k_3}$, \ldots, $B_{k_{l-1}}$,
    \item Finally, $\gamma_{ij}$ goes along $\partial B_j$, ending at $f_j$.
\end{enumerate}
Let $\gamma$ be the concatenation of all $\gamma_{ij}$. Then $\gamma$ is a well-defined sparse-centered curve containing $\partial S$.
\end{claim}

The proof is given in Appendix~\ref{section:proof-of-claim}.


\section{The local inequality implies Lemma~\ref{lemma:good_curve}}

The goal of this section is to formulate an upper bound on the length of each $\gamma_{ij}$ and to show how it implies Lemma~\ref{lemma:good_curve}.
% In order to prove Lemma~\ref{lemma:good_curve}, we need to build sufficiently short parts $\gamma_{ij}$. In this section we formalize what ``sufficiently short'' means, and then show that the inequality we state is enough to establish Lemma~\ref{lemma:good_curve}. % To bound the length of $\gamma_{ij}$, we introduce the following angular notation.

\subsection{Angular notation}

Recall that the edges of the $\mathcal{P}$-tree are directed from parents. Consider these edges as unit vectors and denote by $E$ the set of all these edges.
% While there may exist an edge from the 0-disk and an edge from a 1-disk which equal each other as directed segments, we still distinguish the edges from the source and the edges from 1-disks.
% In this section we denote the vectors from the 0-disk by $u_1$, $u_2$, etc., and the vectors from 1-disks by $v_1$, $v_2$, etc. (the choice of index is completely defined by the context).
For any $2$-disk $D_i$ let $v_i = c_{i+1} - c_i$ and $u_i = c_{i+1} - c$.

We define two functions $\angle(\cdot, \cdot)\colon E^2\to\mathbb{R}$ and $\angccw(\cdot, \cdot)\colon (\mathbb{R}^2\setminus\{c\})^2\to[0, 2\pi)$. Essentially, each of them is a directed angle between two vectors; that is, the angle the first vector needs to be rotated by counterclockwise in order to become the positive scalar multiple of the second one. However, we need to define the range of values for $\angle(\cdot, \cdot)$ and we do it in the following way (here $i$ and $j$ are indices of some $2$-disks).

\begin{enumerate}[label={\{\arabic*\}}]
\item $\angle(u_i, u_j)\in[0, 2\pi)$; \label{rule:uu}
\item $\angle(u_i, v_i)\in(-\pi, \pi)$; \label{rule:uivi} % . Furthermore, a stronger inequality $\angle(u_i, v_i)\in[-2\pi/3, 2\pi/3]$ follows from the fact that every two centers are distance at least $1$ away from each other; hence, all edges from any tree vertex (both incoming and outgoing) are at least $\pi/3$ away from each other;
\item $\angle(v_i, u_i) = -\angle(u_i, v_i)$; \label{rule:viui}
\item $\angle(v_i, u_j) = \angle(v_i, u_i) + \angle(u_i, u_j)$, and $\angle(u_j, v_i) = \angle(u_j, u_i) + \angle(u_i, v_i)$; \label{rule:uivj}
% \item If $u_2$ points at the parent of an edge $v_2$, then for any $u_1$ we have $\angle(u_1, v_2) = \angle(u_1, u_2) + \angle(u_2, v_2)$;
\item $\angle(v_i, v_j) = \angle(v_i, u_i) + \angle(u_i, u_j) + \angle(u_j, v_j)$. \label{rule:vv}
\end{enumerate}

The $\angle(\cdot, \cdot)$ operator is clearly well-defined.

% We will also use a similar operator $\angccw(\cdot, \cdot)$ returning an ordinary counterclockwise angle between two vectors.

% Now we are ready to consider all possible cases.
% % For a subsegment $\mathcal{D}_{ij}$,
% Consider all tree edges connecting consecutive vertices of $\mathcal{D}_{ij}$
% % (that is, $c_ic_{i+1}$, \ldots, $c_{j-1}c_j$),
% and denote by $v_i$ and $v_j$ the two of them that outgo from $1$-disks (namely, we denote $\overrightarrow{c_{i+1}c_i}$ by $v_i$ and $\overrightarrow{c_{j-1}c_j}$ by $v_j$). We also denote $\overrightarrow{cc_{i+1}}$ by $u_i$ and $\overrightarrow{cc_{j-1}}$ by $u_j$. In particular, if $k = 0$, then $u_i = u_j$.

\begin{claim}
\label{lemma:can-construct}

Let $i$ and $j$ be two consecutive $2$-disks and let $k$ be the number of occurrences of the source in $\mathcal{D}_{ij}$. Denote $\angle(v_i, v_j)$ by $\varphi_{ij}$ and $\angccw(c_i, c_j)$ by $\alpha_{ij}$. If $k = 1$, also denote $\angle(u_i, u_j)$ by $\psi_{ij}$. Then we have

\begin{equation*}
\label{eq:curve_bound}
|\gamma_{ij}|\leq\begin{cases}
\displaystyle
\varphi_{ij} + \alpha_{ij}, & \text{when }k = 0, \\
3\psi_{ij} - \frac{2\pi}{3} + 2\varphi_{ij} + \alpha_{ij}, & \text{when }k = 1.
\end{cases}%\tag{*}
\end{equation*}
\end{claim}

% We prove Claim~\ref{lemma:can-construct} in the next section.

If we denote by $\mathcal{R}$ the set of all regions $R_{ij}$, then we obtain \begin{equation}\sum_{R_{ij}\in\mathcal{R}}\varphi_{ij} = \sum_{R_{ij}\in\mathcal{R}}\alpha_{ij} = \sum_{R_{ij}\in\mathcal{R}}\psi_{ij} = 2\pi.\label{eqn:sums-of-angles-are-2pi}
\end{equation}
Indeed, the vectors $u_i$ divide $2\pi$ into angles $\psi_{ij}$, and the rays from $c$ through the centers of $2$-disks divide $2\pi$ into angles $\alpha_{ij}$.

To establish the equality for $\varphi_{ij}$, let $v_{i_1}$, \ldots, $v_{i_m}$ be the vectors outgoing from any $1$-disk, and let $u = u_{i_1} = \ldots = u_{i_m}$ be the vector from the source to it.
We have
\begin{align*}
\angle(v_{i_{m-1}}, v_{i_m}) + \angle(v_{i_m}, u) & \stackrel{\ref{rule:vv}}{=} \big(\angle(v_{i_{m-1}}, u) + \angle(u, u) + \angle(u, v_{i_m})\big) + \angle(v_{i_m}, u) \\ & \stackrel{\ref{rule:viui}}{=} \angle(v_{i_{m-1}}, u).
\end{align*}
Therefore, we obtain
\begin{multline*}
\angle(u, v_{i_1}) + \angle(v_{i_1}, v_{i_2}) + \ldots + \angle(v_{i_{m-1}}, v_{i_m}) + \angle(v_{i_m}, u) \\
= \angle(u, v_{i_1}) + \angle(v_{i_1}, v_{i_2}) + \ldots + \angle(v_{i_{m-2}}, v_{i_{m-1}}) + \angle(v_{i_{m-1}}, u).
\end{multline*}
Thus, by induction,
$$\angle(u, v_{i_1}) + \angle(v_{i_1}, v_{i_2}) + \ldots + \angle(v_{i_{m-1}}, v_{i_m}) + \angle(v_{i_m}, u) = 0.$$

% \begin{multline*}
% \angle(u, v_{i_1}) + \angle(v_{i_1}, v_{i_2}) + \ldots + \angle(v_{i_{m-1}}, v_{i_m}) + \angle(v_{i_m}, u) \\
% \stackrel{\ref{rule:uivj}}{=} \angle(u, v_{i_2}) + \angle(v_{i_2}, v_{i_3}) + \ldots + \angle(v_{i_{m-1}}, v_{i_m}) + \angle(v_{i_m}, u) \\
% = \cdots = \angle(u, v_{i_m}) + \angle(v_{i_m}, u) \stackrel{\ref{rule:viui}}{=} 0.
% \end{multline*}

One can see that after expanding $\sum\varphi_{ij}$ by the definition (see~\ref{rule:vv}) and subtracting the left hand side of the equality above for all $1$-disks, we obtain exactly $\sum\psi_{ij}$. Indeed, after canceling everything out, the only summands that remain are $\angle(u_i, u_j) = \psi_{ij}$.

\begin{proof}[Claim~\ref{lemma:can-construct} implies Lemma~\ref{lemma:good_curve}]
% Concatenating all $\gamma_{ij}$ in the proper order, we obtain the curve $\gamma$. At this moment we have an intermediate condition on the curve $\gamma$: namely, the upper bound for the length of each $\gamma_{ij}$. % Next, we show how to finish the proof. % that this upper bound implies that the total number of disks will be sufficiently small, and the rest of this section will be devoted to the manual construction of~$\gamma_{ij}$.

% \begin{lemma}
% If the inequalities from~(\ref{eq:curve_bound}) hold, then the curve satisfies the assumptions of Lemma~\ref{lemma:good_curve}.
% \end{lemma}

%Let $C_1'$ be the number of non-childfree $1$-disks of $\mathcal{P}$. Clearly, $C_1'\leq C_1$.
%
Denote by $\mathcal{R}_0$ the set of all regions $R_{ij}$, for which the source does not occur in $\mathcal{D}_{ij}$, that is, $\mathcal{D}_{ij}$ consists of three disks. Denote the set of all other regions by $\mathcal{R}_1$. In particular, $\mathcal{R} = \mathcal{R}_0\cup\mathcal{R}_1$ and $\varphi_{ij}\geq\pi/3$ for $R_{ij}\in\mathcal{R}_0$. Also, $|\mathcal{R}_1| = C_1$ and $|\mathcal{R}_0| = C_2 - C_1$.
% If $R = R_{ij}$, denote the angles $\psi_{ij}$, $\varphi_{ij}$, and $\alpha_{ij}$ by $\psi_R$, $\varphi_R$, and $\alpha_R$ respectively.
Let us bound the length of $\gamma$ as the sum of lengths $\gamma_{ij}$ using Claim~\ref{lemma:can-construct}.
%
\begin{align*}
    |\gamma| &\leq \sum_{R_{ij}\in\mathcal{R}_0}(\varphi_{ij} + \alpha_{ij}) + \sum_{R_{ij}\in\mathcal{R}_1}\left(3\psi_{ij} - \frac{2\pi}{3} + 2\varphi_{ij} + \alpha_{ij}\right)  \\
    % &= \sum_{R_{ij}\in\mathcal{R}}(2\varphi_{ij} + \alpha_{ij}) + 3\sum_{R_{ij}\in\mathcal{R}_1}\psi_{ij} - \frac{2\pi}{3}|\mathcal{R}_1| - \sum_{R_{ij}\in\mathcal{R}_0}\varphi_{ij}  \\
    &= 2\sum_{R_{ij}\in\mathcal{R}}\varphi_{ij} + \sum_{R_{ij}\in\mathcal{R}}\alpha_{ij} + 3\sum_{R_{ij}\in\mathcal{R}}\psi_{ij} - \frac{2\pi}{3}\cdot C_1 - \sum_{R_{ij}\in\mathcal{R}_0}\varphi_{ij}  \\
    &\stackrel{\eqref{eqn:sums-of-angles-are-2pi}}{\leq} 2\cdot 2\pi + 2\pi + 3\cdot 2\pi - \frac{2\pi}{3}\cdot C_1 - \frac{\pi}{3}\cdot (C_2 - C_1)  \\
    % &= 12\pi - \frac{2\pi}{3}\cdot C_1 - (C_2 - C_1)\cdot\frac{\pi}{3}  \\
    &= 12\pi - \frac{\pi}{3}(C_1 + C_2).
\end{align*}

The last inequality is equivalent to $C_1 + C_2 + \frac{|\gamma|}{\pi/3}\leq 36$.
\end{proof}

% \section{Proof of Claim~\ref{lemma:can-construct}}

\subsection{Direction jump}

% We introduce some notation in order to bound $|\gamma_{ij}|$.
Instead of proving Claim~\ref{lemma:can-construct}, first we simplify its statement. For that, we introduce the concept of direction jump.

Let the sparse-centered curve $\gamma_{ij}$ be composed of $m$ arcs. Let $w_k^s$ and $w_k^f$ be the unit vectors from the center of the $k\textsuperscript{th}$ arc to its beginning and its end, respectively (for all $k\in[m]$). In particular, $w_1^s = f_i - c_i$ and $w_m^f = f_j - c_j$. By the \emph{direction jump} between the $k\textsuperscript{th}$ and the $(k+1)\textsuperscript{th}$ arcs or the \emph{$k\textsuperscript{th}$ direction jump}, we define $\angccw(w_{k+1}^s, w_k^f)$; see Figure~\ref{fig:direction-jumps}. We denote the $k\textsuperscript{th}$ direction jump by $\dj_k$. The sum of all directed jumps $\sum_{k=1}^{m-1}\dj_k$ is denoted by $\Dj_{ij}$.

% It can be seen on Figure~\ref{fig:direction-jumps} that there may appear a rhombus, whose two opposite angles are $\psi_{ij}$ and a direction jump. In such situations we sometimes replace the corresponding direction jump by $\psi_{ij}$ without any further comments.
For example, on Figure~\ref{fig:direction-jumps} we have $\dj_1 = \dj_3 = \pi/3$, and $\dj_2 = \psi_{ij}$.

If $B_{k_1}$ and $B_{k_2}$ are two disks, we will call the \emph{direction jump between $B_{k_1}$ and $B_{k_2}$} the direction jump between the arcs of these disks which are present in the curve $\gamma_{ij}$. % We will also sometimes call it the direction jump between $D_{k_1}$ and $D_{k_2}$.

\begin{figure}[h!]
    \centering
    \includegraphics[width=.6\textwidth]{pics/direction-jump.mps}
    \caption{Direction jumps}
    \label{fig:direction-jumps}
\end{figure}

\begin{observation}
For the curve $\gamma_{ij}$ the following holds.
% Let $i = k_1 < \ldots < k_m = j$ be the set of indices of disks, whose arcs compose $\gamma_{ij}$.
% Introduce vectors $w_{l}^s$ and $w_{l}^f$ and direction jumps $\dj_l$ as in the definition above. Then 
$$|\gamma_{ij}| = \Dj_{ij} + \alpha_{ij}.$$
\end{observation}

\begin{proof}
% Denote the length of $\gamma_{ij}$ by $\ell$. By definition,
Note that
\begin{align*}
\alpha_{ij} & = \angccw(w_{1}^s, w_{m}^f) \\
& = \sum_{l=1}^{m}\angccw(w_{l}^s, w_{l}^f) - \sum_{l=1}^{m-1}\angccw(w_{l+1}^s, w_{l}^f) \\
& = |\gamma_{ij}| - \sum_{l=1}^{m-1}\dj_{l} = |\gamma_{ij}| - \Dj_{ij},
\end{align*}
which finishes the proof.
\end{proof}

% \begin{lemma}
% Let a sparse-centered curve be composed of $m$ arcs. Suppose that the circles containing its first and last arcs are centered respectively at $c_i$ and $c_j$. Also suppose that the curve starts at $f_i$ and ends at $f_j$. Denote $\angccw(f_i, f_j)$ by $\alpha$, and denote the $l\textsuperscript{th}$ direction jump by $\beta_l$. Then the length of this curve equals $\alpha + \sum_{l=1}^{m-1}\beta_l$.
% \end{lemma}

% \begin{proof}
% We keep track of how $\gamma'(t)$ rotates as $t$ increases (assuming that $\gamma$ is parametrized naturally). The direction of $\gamma'(t)$ rotates with a constant speed along the smooth arcs, and increases by exactly $\alpha$ in total. The direction jumps add extra $\sum\beta_l$.
% \end{proof}

% Thus, instead of bounding the length of the part of the curve inside a region $R$ by $3\psi_R - \frac{2\pi}{3} + 2\varphi_R + \alpha_R$ (and $\varphi_R + \alpha_R$ for $k = 0$), we can bound the sum of all direction jumps by $3\psi_R - \frac{2\pi}{3} + 2\varphi_R$ (respectively, $\varphi_R$) instead.

This observation implies that in order to prove Claim~\ref{lemma:can-construct} it suffices to show the following assertion.

\begin{claim}\label{lemma:can-construct-without-alpha} We have
$$\Dj_{ij}\leq\begin{cases}\varphi_{ij}, & \textrm{if }k = 0, \\ 3\psi_{ij} - \frac{2\pi}{3} + 2\varphi_{ij}, & \textrm{if }k = 1. \end{cases}$$
\end{claim}

\section{Proof of Claim~\ref{lemma:can-construct-without-alpha}}

For simplicity, we use $\psi$, $\varphi$, and $\Dj$ instead of $\psi_{ij}$, $\varphi_{ij}$, and $\Dj_{ij}$ in this section.

\begin{observation}\label{lemma:not-covered-2pi-3}
% If the inequality $\angccw(c_{l-1} - c_l, c_{l+1} - c_l) < 2\pi/3$ holds for $i < l < j$, then $B_l$ is not involved in $\sigma_{ij}$. Conversely,
If a disk $B_l$ is involved in $\sigma_{ij}$ for $i < l < j$, then $\angccw(c_{l-1} - c_l, c_{l+1} - c_l)\geq 2\pi/3$.
\end{observation}

\begin{proof}
Consider any point $p$ of $\partial{B_l}\cap\sigma_{ij}$. As $p$ is not inside $B_{l+1}$, we have $|p - c_{l+1}|\geq 1$, or $\angccw(p - c_l, c_{l+1} - c_l)\geq\pi/3$. Similarly, $\angccw(c_{l-1} - c_l, p - c_l)\geq\pi/3$. Hence, $\angccw(c_{l-1} - c_l, c_{l+1} - c_l)\geq2\pi/3$.
\end{proof}

\subsection{Case $k = 0$}

There are two $2$-disks $D_i$ and $D_j$ and one $1$-disk $D_{i+1} = D_{j-1}$ occurring in $\mathcal{D}_{ij}$.

\begin{enumerate}[label={\bf Case \arabic*: }, wide, labelwidth=!, labelindent=0pt]
%\subsubsection{$k = 0$, $\varphi\leq 2\pi/3$}
%\textbf{Case 1: $k = 0$, $\varphi\leq 2\pi/3$.}
\caseitem{0a}{$B_{i+1} = B_{j-1}$ is not involved in $\sigma_{ij}$}

In this case there is the only direction jump of size $\varphi$. % ; see Figure~\ref{fig:0a_0b}.

% \subsubsection{$k = 0$, $\varphi > 2\pi/3$}
\caseitem{0b}{$B_{i+1} = B_{j-1}$ is involved in $\sigma_{ij}$}

In this case there are two direction jumps, each of them equal to $\pi/3$. Due to Observation~\ref{lemma:not-covered-2pi-3}, $\varphi\geq 2\pi/3$, which finishes the proof.

% \begin{figure}[h!]
%     \centering
%     \begin{subfigure}{.4\textwidth}
%     \includegraphics[width=\textwidth]{pics/case-11.mps}
%     % \caption{both angles are at most $\pi/3$}
%     \end{subfigure}
%     \begin{subfigure}{.5\textwidth}
%     \includegraphics[width=\textwidth]{pics/case-12.mps}
%     % \caption{only one angle is at most $\pi/3$}
%     \end{subfigure}
%     \caption{Cases 0a and 0b}
%     \label{fig:0a_0b}
% \end{figure}

% \subsubsection{$k = 1$: general observations}
% \caseitem{1}{$k = 1$, general observations}
\subsection{Case $k = 1$, general observations}


There are two $2$-disks $D_i$ and $D_j$, two $1$-disks $D_{i+1}$ and $D_{j-1}$, and one $0$-disk $D$ occurring in $\mathcal{D}_{ij}$.

\begin{observation}\label{lemma:direction-jumps-are-bounded}
The direction jump between disks $B_{i_1}$ and $B_{i_2}$ does not exceed $(i_2 - i_1)\frac{\pi}{3}$.
\end{observation}

\begin{proof}
Let $p$ be the point of the direction jump from $B_{i_1}$ to $B_{i_2}$. Then, as for every $l$ such that $i_1 < l < i_2$ we have $|p - c_l|\geq 1$, the vertex $p$ is always at the smallest angle of any triangle $c_lc_{l+1}p$ for $i_1\leq l < i_2$. Since the angle about $p$ is at most $\pi/3$ in every such triangle, summing these inequalities gives us the required bound.
\end{proof}

\begin{observation}
If $\psi\geq\pi$, then Claim~\ref{lemma:can-construct-without-alpha} holds.
\end{observation}

\begin{proof}
Note that $\angle(u_i, v_i)\in\left[-\frac{2\pi}{3}, \frac{2\pi}{3}\right]$. Indeed, otherwise the angle between $c_{i+1} - c_i$ and $c_{i+1} - c$ is less than $\pi/3$, which implies that $|c_i - c| < 1$. Similarly, $\angle(u_j, v_j)\in\left[-\frac{2\pi}{3}, \frac{2\pi}{3}\right]$.

But then
$$\varphi \stackrel{\ref{rule:vv}}{=} \angle(v_i, u_i) + \angle(u_i, u_j) + \angle(u_j, v_j) \geq -\frac{2\pi}{3} + \psi - \frac{2\pi}{3} \geq -\frac{\pi}{3}.$$

Thus,
$3\psi - \frac{2\pi}{3} + 2\varphi\geq \frac{5\pi}{3}.$

On the other hand, $\Dj\leq\frac{4\pi}{3}$. Indeed, if $i = i_1$, $i_2$, \ldots, $i_m = j$ are all involved in $\sigma_{ij}$ disks, then, by Observation~\ref{lemma:direction-jumps-are-bounded}, we have $$\Dj\leq(i_2 - i_1)\frac{\pi}{3} + \ldots + (i_m - i_{m-1})\frac{\pi}{3} = \frac{4\pi}{3},$$
which concludes the proof.
\end{proof}

From now on, we assume $\psi < \pi$.

\begin{observation} \label{lemma:inside-the-rhombus}
If $c'$ is the point such that $c_{i+1}cc_{j-1}c'$ is a parallelogram, then $c'\in R_{ij}$. Let $t$ be any point such that $|t - c_{i+1}|\geq 1$ and $|t - c_{j-1}| \geq 1$. Then $t$ cannot be inside the rhombus $c_{i+1}cc_{j-1}c'$.
\end{observation}

\begin{proof}
% It is evident that $c'\in R_{ij}$; see Figure~\ref{fig:phi-and-psi}. Indeed, t
The triangle $c_{i+1}cc'$ lies in $B_{i+1}$, and the triangle $c_{j-1}cc'$ lies in $B_{j-1}$. If $c'\notin R_{ij}$, then there is a unit segment or a ray that is a side of $R_{ij}$ and that separates $c'$ from $c$ in this region. % which is obviously not the case.
It is obvious that it cannot be any of the rays, as both of them belong to the line containing $c$. Similarly, it can be neither $cc_{i+1}$ or $cc_{j-1}$. Hence, the only remaining options are segments $c_ic_{i+1}$ and $c_jc_{j-1}$. Suppose that, say, the segment $c_ic_{i+1}$ intersects the segment $cc'$.
This means that the point $|c_{j-1} - c_i| < 1$, because in the triangle $c_ic_{i+1}c_{j-1}$ the angle about $c_{j-1}$ is greater than the angle about $c_{i+1}$, and $|c_{i+1} - c_i| = 1$.
Similarly, $c_jc_{j-1}$ cannot separate $c$ from $c'$. Thus, $c'\in R_{ij}$.

Suppose that $t$ belongs to one of the triangles $cc_{i+1}c'$ and $cc_{j-1}c'$, say, the first one. But since $|c_{i+1} - c| = |c_{i+1} - c'| = 1$, any other point of between $c$ and $c'$ is strictly less than $1$ away from $c_{i+1}$; therefore, so is point $t$.
\end{proof}

\begin{observation} \label{lemma:psi-plus-phi} We have
\begin{align*}
\psi + \varphi & = \angle(u_i, v_j) + \angle(v_i, u_j) \\
& = \angccw(c_i - c_{i+1}, c' - c_{i+1}) + \angccw(c' - c_{j-1}, c_j - c_{j-1}).    
\end{align*}

\end{observation}

\begin{proof}
Note that neither $c_i$ nor $c_j$ is inside the rhombus $c_{i+1}cc_{j-1}c'$, according to Observation~\ref{lemma:inside-the-rhombus}. Thus, $\angle(v_i, u_j)\geq 0$ and $\angle(u_i, v_j)\geq 0$. Hence,
\begin{align*}
\psi + \varphi & = \angle(v_i, v_j) + \angle(u_i, u_j) \\
&\stackrel{\ref{rule:vv}}{=} \big(\angle(v_i, u_i) + \angle(u_i, u_j) + \angle(u_j, v_j)\big) + \angle(u_i, u_j) \\
&= \big(\angle(v_i, u_i) + \angle(u_i, u_j)\big) + \big(\angle(u_i, u_j) + \angle(u_j, v_j)\big) \\
&\stackrel{\ref{rule:uivj}}{=} \angle(v_i, u_j) + \angle(u_i, v_j) \\
&= \angccw(c_i - c_{i+1}, c' - c_{i+1}) + \angccw(c' - c_{j-1}, c_j - c_{j-1}),
\end{align*}
which finishes the proof.
\end{proof}

\begin{figure}[h!]
    \centering
    \begin{subfigure}[t]{.4\textwidth}
    \includegraphics[width=\textwidth]{pics/phi-and-psi.mps}
    \caption{Curve from Obsevation~\ref{lemma:obvious-stuff-about-psi}}
    \label{fig:phi-and-psi}
    \end{subfigure}
    \begin{subfigure}[t]{.4\textwidth}
    \includegraphics[width=\textwidth]{pics/only-middle-involved.mps}
    \caption{Curve from Obsevation~\ref{lemma:involved-source}}
    \label{fig:only-middle-involved}
    \end{subfigure}
    \caption{}
\end{figure}

\begin{observation} \label{lemma:obvious-stuff-about-psi}
$\psi + \varphi\geq \pi/3$ and $\psi\geq \pi/3$.
\end{observation}

\begin{proof}
Since $|c_{i+1} - c_{j-1}| \geq 1$, we have $\psi\geq\pi/3$. Consider the sparse-centered curve that starts at $c_i$, follows the perimeter of $B_{i+1}$ counterclockwise until it reaches $c'$, then switches to the perimeter of $B_{j-1}$ until it reaches $c_j$; see Figure~\ref{fig:phi-and-psi}. Since it satisfies the assumption of Lemma \ref{lemma:master}, its length is at least $\pi/3$. On the other hand, according to Observation~\ref{lemma:psi-plus-phi}, its length is exactly $\psi + \varphi$.
\end{proof}

\begin{observation}
If $B$ is involved in $\sigma_{ij}$, then $\varphi\geq 0$. \label{lemma:involved-source}
\end{observation}

\begin{proof}
It follows from the constraints that the arc of $B$ that goes from $c_{i+1}$ to $c_{j-1}$ counterclockwise and which has length $\psi$ is not completely covered by other disks; in particular, it is not fully covered by $B_i$ and $B_j$.
Note that if $\angle(u_i, v_i) > 0$, then $B_i$ covers the arc of length $\angle(u_i, v_i)$ of $B$, starting at $c_{i+1}$. In other words, $B_i$ covers an arc of length $\max(\angle(u_i, v_i), 0)$, starting at $c_{i+1}$. Similarly, $B_j$ covers an arc of length $\max(\angle(v_j, u_j), 0)$, ending at $c_{j-1}$, see Figure~\ref{fig:only-middle-involved}.

% \begin{figure}[h!]
%     \centering
%     \includegraphics[width=.6\textwidth]{pics/only-middle-involved.mps}
%     \caption{$I = \{B_{i+2}\}$}
%     \label{fig:only-middle-involved}
% \end{figure}

Since $B_{i+2}$ is involved in $\sigma_{ij}$, we have
\begin{align*}
    \psi & \geq\max(\angle(u_i, v_i), 0) + \max(\angle(v_j, u_j), 0) \\
    & \geq\angle(u_i, v_i) + \angle(v_j, u_j) \\
    & \stackrel{\ref{rule:uivi}}{=} \angle(u_i, u_j) + \angle(v_j, v_i) = \psi - \varphi,
\end{align*}
which implies that $\varphi\geq 0$.\end{proof}

\begin{corollary}\label{lemma:middle-involved-side-angles}
If $B$ is involved in $\sigma_{ij}$, then
$$\angccw(c_i - c_{i+1}, c - c_{i+1}) + \angccw(c - c_{j-1}, c_j - c_{j-1})\leq 3\psi - \frac{2\pi}{3} + 2\varphi.$$
\end{corollary}

\begin{proof}
According to Observation~\ref{lemma:involved-source}, $\varphi\geq 0$. Also, due to Observation~\ref{lemma:not-covered-2pi-3}, $\psi\geq2\pi/3$. But then
\begin{align*}
    & \angccw(c_i - c_{i+1}, c - c_{i+1}) + \angccw(c - c_{j-1}, c_j - c_{j-1}) \\
    & = (\pi - \angle(u_i, v_i)) + (\pi - \angle(v_j, u_j)) \\
    & = 2\pi - \psi + \varphi \\
    & \leq 2\pi - \psi + \varphi + 4\left(\psi - \frac{2\pi}{3}\right) + \varphi = 3\psi - \frac{2\pi}{3} + 2\varphi.
\end{align*}
\end{proof}

\subsection{Case $k = 1$, finishing the proof}

Let $I$ be the set of disks from $\{B_{i+1}, B, B_{j-1}\}$ which are involved in $\sigma_{ij}$.

\caseitem[subsec:case-000]{1a}{$I = \varnothing$}

There is the only direction jump, and it is between the disks $B_i$ and $B_j$. Denote this direction jump by $\dj$. As $\psi\geq\pi/3$ by Observation~\ref{lemma:obvious-stuff-about-psi}, it suffices to prove that $\dj\leq 2\psi - \frac{\pi}{3} + 2\varphi$.

If $\psi + \varphi\geq 2\pi/3$, then, obviously, $\dj\leq \pi = 2\cdot\frac{2\pi}{3} - \frac{\pi}{3}\leq 2(\psi + \varphi) - \frac{\pi}{3}$. Hence, one may safely assume that $\psi + \varphi\leq 2\pi/3$, which together with the inequality $\psi\geq\pi/3$ gives us

\begin{equation}
\label{eq:straighten}
\psi \geq \frac{\pi}{3}\geq \psi + \varphi - \frac{\pi}{3}\geq \varphi.
\end{equation}

By the triangle inequality, we obtain
$$2\sin\frac{\dj}{2} = |c_i - c_j| = |u_i + v_i - u_j - v_j|\leq |u_i - u_j| + |v_i - v_j| = 2\sin\frac{\psi}{2} + 2\sin\frac{\varphi}{2}.$$ Here we use that in a triangle with two unit sides and angle $\beta$ between them the third side has length $2\sin(\beta/2)$.

If $s = \psi + \varphi\in\left[\frac{\pi}{3}, \frac{2\pi}{3}\right]$ is fixed, then the right hand side is maximized when $|\psi - \varphi|$ is minimized. Indeed, if $z = e^{i\psi/2} + e^{i\varphi/2}$, then $\arg{z} = s/4$; therefore maximizing $\Im{z}$ means maximizing $|z|$, or minimizing the angle between $e^{i\psi/2}$ and $e^{i\varphi/2}$. According to \eqref{eq:straighten}, it is equivalent to setting $\psi=\pi/3$ and $\varphi = s - \pi/3$. Then $$2\sin\frac{\dj}{2}\leq 2\left(\frac{1}{2} + \sin\left(\frac{s}{2} - \frac{\pi}{6}\right)\right).$$ Therefore it is enough to verify that $$\frac{1}{2} + \sin\left(\frac{s}{2} - \frac{\pi}{6}\right)\leq\sin\left(s - \frac{\pi}{6}\right).$$
The last inequality holds because $$\frac{\partial^2}{\partial s^2}\left(\sin\left(s - \frac{\pi}{6}\right) - \sin\left(\frac{s}{2} - \frac{\pi}{6}\right)\right) = \frac{1}{4}\sin\left(\frac{s}{2} - \frac{\pi}{6}\right) - \sin\left(s - \frac{\pi}{6}\right) < 0$$
for $\pi/3\leq s\leq 2\pi/3$ and
$$\frac{1}{2} = \sin\left(s - \frac{\pi}{6}\right) - \sin\left(\frac{s}{2} - \frac{\pi}{6}\right)$$
for $s\in\{\pi/3, 2\pi/3\}$.

\caseitem[subsec:case-001]{1b}{$I = \{B_{i+1}\}$}

There are two direction jumps: between $B_i$ and $B_{i+1}$ of size $\pi/3$ and between $B_{i+1}$ and $B_j$.

Let $t$ be the point of the latter direction jump.
By Observation~\ref{lemma:inside-the-rhombus}, the point $t$ cannot lie inside the rhombus $cc_{i+1}c'c_{j-1}$, and therefore, the point $c'$ lies in the pentagon $c_{j}c_{j-1}cc_{i+1}t$.
Hence, if we apply Lemma~\ref{lemma:master} to the curve following the perimeter of $B_{i+1}$ from $t$ to $c'$ and then passing the perimeter of $B_{j-1}$ from $c'$ to $c_j$, we obtain
\begin{equation}
\angccw(t - c_{i+1}, c' - c_{i+1}) + \angccw(c' - c_{j-1}, c_j - c_{j-1})\geq\frac{\pi}{3}. \label{eqn:with-t}
\end{equation}

Since $B_{i+1}$ is involved in $\sigma_{ij}$, we have
\begin{align}
\angccw(t - c_{i+1}, c' - c_{i+1}) & = \angccw(c_i - c_{i+1}, c' - c_{i+1}) - \angccw(c_i - c_{i+1}, t - c_{i+1}) \notag \\ 
& \leq \angccw(c_i - c_{i+1}, c' - c_{i+1}) - \pi / 3 \label{eqn:rotate-equilateral} \\ 
& = \angle(v_i, u_j) - \pi / 3. \notag
\end{align}

\begin{figure}[h!]
    \centering
    \begin{subfigure}[t]{.4\textwidth}
    \includegraphics[width=\textwidth]{pics/huge-formula-pic.mps}
    \caption{Case~\ref{subsec:case-001}}
    \label{fig:huge-formula}
    \end{subfigure}
    \begin{subfigure}[t]{.4\textwidth}
    \includegraphics[width=\textwidth]{pics/case-3.mps}
    \caption{Case~\ref{subsec:case-101}}
    \label{fig:case-1c}
    \end{subfigure}
    \caption{}
\end{figure}

We are ready to bound $\Dj$. One may refer to Figure~\ref{fig:huge-formula} to follow the explanation.
\begin{align*}
\Dj & = \frac{\pi}{3} + \angccw(t - c_j, t - c_{i+1}) \\
& = \frac{\pi}{3} + \pi - \angccw(t - c_{i+1}, c_j - t)\\
& = \frac{4\pi}{3} - \angccw(t - c_{i+1}, c' - c_{i+1}) + \angccw(c' - c_{j-1}, c' - c_{i+1}) \\ & - \angccw(c' - c_{j-1}, c_j - c_{j-1}) - \angccw(c_j - c_{j-1}, c_j - t).
\end{align*}
% \begin{align*}
% & \leq \pi + \psi - (\angccw(t - c_{i+1}, c' - c_{i+1}) + \angccw(c' - c_{j-1}, c_j - c_{j-1})) \\
% & \stackrel{\eqref{eqn:with-t}}{\leq} \psi + 2(\angccw(t - c_{i+1}, c' - c_{i+1}) + \angle(u_i, v_j)) \\
% & \stackrel{\eqref{eqn:rotate-equilateral}}{\leq} -\frac{2\pi}{3} + \psi + 2(\angle(v_i, u_j) + \angle(u_i, v_j)) \\
% & = 3\psi - \frac{2\pi}{3} + 2\varphi.
% \end{align*}

Here, the second equality follows from replacing $\angccw(t - c_j, t - c_{i+1})$ by its adjacent angle. Then, since $\angccw(t - c_{i+1}, c_j - t)$ is, by definition, the angle we need to rotate the vector $t - c_{i+1}$ by in order to obtain the vector $c_j - t$; we may first rotate it counterclockwise until we obtain $c' - c_{i + 1}$, then clockwise until $c' - c_{j-1}$, then counterclockwise until $c_j - c_{j-1}$, and, finally, counterclockwise until $c_j - t$. This implies the last equality.

We continue bounding $\Dj$.
\begin{align*}
\Dj & \leq \pi + \psi - (\angccw(t - c_{i+1}, c' - c_{i+1}) + \angccw(c' - c_{j-1}, c_j - c_{j-1})) \\
& \stackrel{\eqref{eqn:with-t}}{\leq} \psi + 2(\angccw(t - c_{i+1}, c' - c_{i+1}) + \angle(u_i, v_j)) \\
& \stackrel{\eqref{eqn:rotate-equilateral}}{\leq} -\frac{2\pi}{3} + \psi + 2(\angle(v_i, u_j) + \angle(u_i, v_j)) \\
& = 3\psi - \frac{2\pi}{3} + 2\varphi.
\end{align*}

Here the first inequality is obtained after applying $\angccw(c' - c_{j-1}, c' - c_{i+1}) = \psi$ and $\angccw(c_j - c_{j-1}, c_j - t)\geq\pi/3$, which follows from the fact that $|t - c_{j-1}|\geq 1$. Due to Observation~\ref{lemma:psi-plus-phi}, the last equality holds.

\caseitem[subsec:case-100]{1b'}{$I = \{B_{j-1}\}$}

This case is similar to the previous one.

\caseitem[subsec:case-101]{1c}{$I = \{B_{i+1}, B_{j-1}\}$}

There are three direction jumps: between $B_i$ and $B_{i+1}$ of size $\pi/3$, between $B_{i+1}$ and $B_{j-1}$ of size $\psi$, and between $B_{j-1}$ and $B_j$ of size $\pi/3$.

According to Observation~\ref{lemma:psi-plus-phi}, $\psi + \varphi = \angle(u_i, v_j) + \angle(v_i, u_j) \ge \pi/3 + \pi/3$, see Figure~\ref{fig:case-1c}. Thus,
$$\Dj = \frac{2\pi}{3} + \psi \leq 3\psi - \frac{2\pi}{3} + 2\varphi,$$
which completes the proof of this case.

% \begin{figure}[h!]
%     \centering
%     \begin{subfigure}[t]{.3\textwidth}
%     \includegraphics[width=\textwidth]{pics/case-1.mps}
%     \caption{$I = \varnothing$}
%     \label{fig:case-1a}
%     \end{subfigure}
%     \begin{subfigure}[t]{.3\textwidth}
%     \includegraphics[width=\textwidth]{pics/case-2.mps}
%     \caption{$I = \{B_{i+1}\}$}
%     \label{fig:case-1b}
%     \end{subfigure}
%     \begin{subfigure}[t]{.3\textwidth}
%     \includegraphics[width=\textwidth]{pics/case-3.mps}
%     \caption{$I = \{B_{i+1}, B_{j-1}\}$}
%     \label{fig:case-1c}
%     \end{subfigure}
%     \caption{$k = 1$, $B\notin I$}
%     % \label{fig:my_label}
% \end{figure}

\caseitem[subsec:case-010]{1d}{$I = \{B\}$}

There are two direction jumps: between $B_i$ and $B$ and between $B$ and $B_j$.

By the definition, the direction jumps are exactly $\angccw(c_i - c_{i+1}, c - c_{i+1})$ and $\angccw(c - c_{j-1}, c_j - c_{j-1})$. Hence, Corollary~\ref{lemma:middle-involved-side-angles} finishes the proof.

\caseitem[subsec:case-011]{1e}{$I = \{B_{i+1}, B\}$}

There are three direction jumps: between $B_i$ and $B_{i+1}$ of size $\pi/3$, between $B_{i+1}$ and $B_{i+2}$ of size $\pi/3$, and between $B_{i+2}$ and $B_j$. According to Observation~\ref{lemma:not-covered-2pi-3}, we have $\angccw(c - c_{j-1}, c_j - c_{j-1})\geq2\pi/3$, and thus, the sum of first two direction jumps is no more than $\angccw(c_i - c_{i+1}, c_{i+2} - c_{i+1})$; hence,
$$\Dj\leq \angccw(c_i - c_{i+1}, c - c_{i+1}) + \angccw(c - c_{j-1}, c_j - c_{j-1}).$$
By Corollary~\ref{lemma:middle-involved-side-angles}, this does not exceed $3\psi - \frac{2\pi}{3} + 2\varphi$.

% \begin{figure}[h!]
%     \centering
%     \begin{subfigure}{.45\textwidth}
%     \includegraphics[width=\textwidth]{pics/case-4.mps}
%     \caption{both angles are at most $\pi/3$}
%     \end{subfigure}
%     \begin{subfigure}{.45\textwidth}
%     \includegraphics[width=\textwidth]{pics/case-5.mps}
%     \caption{only one angle is at most $\pi/3$. Note that the real boundary is not connected here}
%     \end{subfigure}
%     \caption{$k = 1$, $\psi > 2\pi/3$}
%     % \label{fig:my_label}
% \end{figure}

\caseitem[subsec:case-110]{1e'}{$I = \{B, B_{j-1}\}$}

This case is similar to the previous one.

\caseitem[subsec:case-111]{1f}{$I = \{B_{i+1}, B, B_{j-1}\}$}

There are four direction jumps, all of size $\pi/3$. % Hence $\Dj = 4\pi/3$, which does not exceed $\angccw(c_i - c_{i+1}, c_{i+2} - c_{i+1}) + \angccw(c_{j-2} - c_{j-1}, c_j - c_{j-1})$, thus Corollary~\ref{lemma:middle-involved-side-angles} finishes the proof.

By Observation~\ref{lemma:not-covered-2pi-3} we have $\psi\geq2\pi/3$, and by Observation~\ref{lemma:involved-source} we have $\varphi\geq 0$. Then

$$3\psi - \frac{2\pi}{3} + 2\varphi\geq 2\pi - \frac{2\pi}{3} = \frac{4\pi}{3} = \Dj.$$

% \begin{figure}[h!]
%     \centering
%     \begin{subfigure}{.45\textwidth}
%     \includegraphics[width=\textwidth]{pics/case-7.mps}
%     \caption{ordinary case}
%     \end{subfigure}
%     \begin{subfigure}{.45\textwidth}
%     \includegraphics[width=\textwidth]{pics/case-6.mps}
%     \caption{a possible case where the actual boundary is not connected}
%     \end{subfigure}
%     \caption{$k = 1$, $I = \{B_{i+1}, B, B_{j-1}\}$}
%     % \label{fig:my_label}
% \end{figure}

\end{enumerate}

\section{Discussion}

We have shown that $f(3) = 37$, where $f(n)$ is the maximum possible number of disks in a packing of kissing radius $n$. In other words, triangular lattice provides the optimal size of the packing. It is not known whether $f(4) = f(3) + 24 = 61$ or not.

% Lemma~\ref{lemma:master} can be generalized on arbitrary distances.

% \begin{lemma}
% Let $\gamma$ be a sparse-centered curve with endpoints $a$ and $b$. If $d = |a - b|$, then
% $$|\gamma|\geq \lfloor d\rfloor\cdot\frac{\pi}{3} + 2\arcsin\frac{\{d\}}{2},$$
% where $\{x\} = x - \lfloor x\rfloor$. Moreover, this inequality is sharp.
% \end{lemma}

% \begin{proof}
% If $d\leq 1$, then the proof is analogous to the one of Lemma~\ref{lemma:master}. Otherwise, by continuity, $\gamma$ can be split into two curves $\gamma_1$ and $\gamma_2$ where the endpoints of $\gamma_1$ are $d - 1$ from each other. Then, by the triangle inequality, the endpoints of $\gamma_2$ are at least $1$ from each other. By induction on $\lfloor d\rfloor$, we obtain that
% $$|\gamma| = |\gamma_1| + |\gamma_2|\geq \lfloor d - 1\rfloor\cdot\frac{\pi}{3} + 2\arcsin\frac{\{d - 1\}}{2} + \frac{\pi}{3} = \lfloor d\rfloor\cdot\frac{\pi}{3} + 2\arcsin\frac{\{d\}}{2}.$$

% To show that the inequality is sharp, pick $c = \lfloor d\rfloor + 1$ points on a line $m$ with distance $1$ between every adjacent pair. Let $\sigma$ be the boundary of unit radius balls centered at these points. The line $m$ divides $\sigma$ into two equal halves, let $n$ be the axis of symmetry of each of them. Let $\gamma$ be a curve lying on $\sigma$ and symmetrical about $n$. We claim that it has the aforementioned length. Again, it can be shown by induction on $\lfloor d\rfloor$. If $d < 1$, then it can be verified by direct computation. Otherwise, let $a$ be the endpoint of $\gamma$ at distance $1$ from $p_1$. Let $b$ be the other endpoint of $\gamma$, and let $b'$ be such point that $b - b' = p_c - p_{c - 1}$.

% \begin{figure}
%     \centering
%     \includegraphics{pics/general-master-lemma.mps}
%     \caption{An optimal curve $\gamma$}
%     \label{fig:my_label}
% \end{figure}

% Since $a$ and $b$ are symmetrical about $n$, and $b - b' = p_c - p_{c-1}$, points $a$, $b'$ and $b$ belong to the same line. Moreover, $b'$ lies between $a$ and $b$, so $|b' - a| = |b - a| - |b - b'| = d - 1$. This implies that the part of $\gamma$ between $a$ and $b'$ is, by induction, of length $\lfloor d - 1\rfloor\cdot\frac{\pi}{3} + 2\arcsin\frac{\{d\}}{2}$. On the other hand, the part of $\gamma$ between $\gamma'$ and $\gamma$ has length $\pi/3$. This proves the induction step.
% \end{proof}

% $$\Dj = \frac{2\pi}{3} + \psi \leq 3\psi - \frac{2\pi}{3} + 2\varphi,$$

\bibliographystyle{unsrt}
\bibliography{three-layers}

\newpage

\appendix
\begin{appendices}

\section{Proof of Claim~\ref{lemma:can-construct-gamma-ij}}\label{section:proof-of-claim}

Before proving Claim~\ref{lemma:can-construct-gamma-ij}, we first state some properties of Delaunay triangulations, which we then use.

\subsection{Properties of Delaunay triangulations}

Recall that a Delaunay triangulation of a finite set of points is a triangulation where, for each triangle $pqr$ and any other point $s$ of the set, $s$ does not lie inside the circumcircle of $pqr$. If a Delaunay triangulation is fixed, we call any of its triangles just a \emph{Delaunay triangle}. In particular, if $pqr$ and $qps$ are two Delaunay triangles, then $\angle prq + \angle qsp\leq\pi$. For any triangle $\Delta$ denote by $R(\Delta)$ the circumradius of $\Delta$.

\begin{lemma}\label{lemma:no-obtuse-in-delaunay}
Let $pqrs$ be a convex quadrilateral. Let $\triangle pqr$ and $\triangle rsp$ be the faces of its Delaunay triangulation. If $R(\triangle pqr) < 1 \leq R(\triangle rsp)$, then $\angle rsp < \pi/2$.
\end{lemma}

\begin{proof}
We argue by contradiction. Suppose that $\angle rsp \geq \pi/2$. Then, since we have a Delaunay triangulation, $\angle pqr + \angle rsp \leq \pi$, or, equivalently, $\angle pqr \leq \pi - \angle rsp \leq \pi/2$. Thus, $\sin\angle pqr \leq \sin\angle rsp$.

Then, by the law of sines, $$2R(\triangle pqr) = \frac{|p - r|}{\sin\angle pqr} \geq \frac{|p - r|}{\sin\angle rsp} = 2R(\triangle rsp),$$ which contradicts the assumption.
\end{proof}

Note that the essential condition in this lemma is $R(\triangle pqr) < R(\triangle rsp)$; however, we strenghtened it by separating both sides with $1$ for more clear applications in the future.

\begin{lemma}\label{lemma:delaunay-sorted-process}
Given a set $S$ of $m$ points on the plane, consider all $\binom{m}{3}$ triangles with vertices among these points in the nondecreasing order of the circumradius. Start with $\mathcal{F} = \varnothing$, and at each step, if the interior of the current triangle does not contain any of the points and does not intersect anything from $\mathcal{F}$, then add the considered triangle to $\mathcal{F}$.

Then the set $\mathcal{F}$ at the end of the process is the set of faces of some Delaunay triangulation of $S$.
\end{lemma}

\textbf{Remark.} Some properties of Delaunay triangulations regarding the circumradii were already discovered, for example, in~\cite{Musin1997Delaunay}.

\begin{proof}
It is obvious that at the end $\mathcal{F}$ will be a triangulation of $S$. Indeed, suppose that $\cup\mathcal{F}\neq\mathrm{conv}\,S$. This means that there is a point $p\in\cup\mathcal{F}\setminus\mathrm{conv}\,S$, and we may assume that $p$ does not belong to any segment between any two points of $S$. Then there is at least one way to finish the triangulation, hence the set of all triangles containing $p$ and whose interiors do not intersect $\cup\mathcal{F}$ is not empty. Therefore, we should have added to $\mathcal{F}$ any of such triangles with minimum circumradius.

If the final triangulation is not Delaunay, then it is possible to do a ``flip''; that is, there is a convex quadrilateral $pqrs$ that $\triangle{pqr}\in\mathcal{F}$ and $\triangle{rsp}\in\mathcal{F}$, and also $\angle{pqr} + \angle{rsp} > \pi$.

\begin{figure}[h!]
    \centering
    \includegraphics{pics/iterative-delaunay.mps}
    \caption{Lemma~\ref{lemma:delaunay-sorted-process}}
    \label{fig:delaunay-flip}
\end{figure}

Note that no other point belongs to the quadrilateral $pqrs$, even to its boundary. Indeed, if, say, there is a point $t\in [p, q]$, then we should have considered $\triangle ptr$ or $\triangle qtr$ before $\triangle pqr$.

Without loss of generality, assume that $R(\triangle{pqr})\leq R(\triangle{rsp})$. Let $\omega$ be the circumcircle of $\triangle{pqr}$. Since $\angle{pqr} + \angle{rsp} > \pi$, we know that $s$ lies inside $\omega$. This implies that $\angle{rsq} > \angle{rpq}$ and $\angle{qsp} > \angle{qrp}$. Since $\angle{rsq} + \angle{qsp} = \angle{rsp} < \pi$, at least one of $\angle{rsq}$ and $\angle{qsp}$ is acute. If $\angle{rsq} < \pi/2$ then $\sin\angle{rsq} > \sin\angle{rpq}$, and, by the law of sines,
$$R(\triangle{rsq}) = R(\triangle{rpq})\cdot\frac{\sin\angle{rpq}}{\sin\angle{rsq}} < R(\triangle{rpq})\leq R(\triangle{rsp}).$$
Then we should have considered $\triangle{rsq}$ before any of $\triangle{pqr}$ and $\triangle{rsp}$ and added it to $\mathcal{F}$.

The case when $\angle{qsp} < \pi/2$ is analogous.
\end{proof}

We also remind that a \emph{Voronoi diagram} of a set $S$ of points is the division of the plane into (possibly unbounded) regions, which are also called \emph{cells}, and the region corresponding to a point $p\in S$ is defined as
$$\{x\in\mathbb{R}^2\,\colon\,\forall q\in S\quad |x - p|\leq |x - q|\}.$$
It is clear that each cell is a polyhedron, that is, an intersection of some halfplanes. It is also known that Voronoi diagram is dual to Delaunay triangulation in a sense that two points are connected by an edge in the Delaunay triangulation if and only if their corresponding Voronoi cells share a side, with some minor nuances.

In particular, a set of points may have several Delaunay triangulations, if there is a circle containing more than three points of $S$ on its boundary and no points from $S$ inside. In this case the points on this circle may be triangulated arbitrarily, and the Voronoi cells of any of them contains the center of this circle. So it would be formally correct to say that if two Voronoi cells share a side, then the corresponding points are connected in every Delaunay triangulation, and if two Voronoi cells share a point, then the corresponding points may or may not be connected in a Delaunay triangulation.

We will use the duality in the following form: if for some points $p$ and $q$ from $S$ and $x$ from $\mathbb{R}^2$ we have
$$\forall r\in S\setminus\{p, q\}\quad |x - p| = |x - q| < |x - r|,$$
then $p$ and $q$ are connected in any Delaunay triangulation of $S$.

\subsection{Proof of Claim~\ref{lemma:can-construct-gamma-ij}}

Recall that we are given a packing $\mathcal{P}$, and $(c_1, \ldots, c_n)$ is the cyclic sequence of all centers of disks from $\mathcal{P}$, in the order of traversal. The corresponding disks of unit diameter are denoted by $D_i$, and $B_i$ is the open disk with unit radius, centered at $c_i$. We say that $\mathcal{D}_{ij} = (D_i, \ldots, D_j)$ is a subsegment, if $D_i$ and $D_j$ are two consecutive $2$-disks. For any subsegment $\mathcal{D}_{ij}$ we denote by $R_{ij}$ the region bounded by segments $[c_i, c_{i+1}]$, \ldots, $[c_{j-1}, c_j]$, and two rays going from $c_i$ and $c_j$ in the direction from the origin, which is also the center of the $0$-disk; see Figure~\ref{fig:regions} on page~\pageref{fig:division-into-regions}. The union of all $B_i$ is denoted by $S$, and $\partial{S}\cap R_{ij}$ is denoted by $\sigma_{ij}$. If $D_k\in\mathcal{D}_{ij}$ and $\partial{B_k}\cap\sigma_{ij}\neq\varnothing$, we say that $B_k$ is \emph{involved} in $\sigma_{ij}$.

We want to show that it is possible to remove some points from the sequence $(c_1, \ldots, c_n)$, so that, if we construct the curve $\gamma$ on the remaining points, it will be valid, in particular, in terms of going counterclockwise.
% Informally, if we just try and build the curve on the initial sequence containing the whole traversal, allowing it to go clockwise, and then exclude the points corresponding to clockwise arcs one by one, we would eventually achieve the goal.
% Instead of dealing with this removal process, which is hard, we will use Delaunay triangulation, as it is essentially the same and is easier and nicer to work with. More specifically, excluding a point $q$ so that the sequence transforms like $(\ldots, p, q, r, \ldots)\to(\ldots, p, r, \ldots)$ will correspond to $\triangle pqr$ being a special face of some Delaunay triangulation.

Let $\mathcal{F}$ be the Delaunay triangulation of the set of points $\{c_1, \ldots, c_n\}$, and let $\{F_1, \ldots\}$ be the set of its triangular faces.
% We will call a triangle $\Delta$ a \emph{Delaunay triangle}, if $\Delta = F_i$ for some $i$.
It can be seen that $\mathcal{F}$ contains an edge between points $c_i$ and $c_{i+1}$. Indeed, if we consider the Voronoi diagram of the set of disks $\{c_1, \ldots, c_n\}$, then, since disks $D_i$ and $D_{i+1}$ touch each other, their common point lies on the common edge of Voronoi cells corresponding to them, as this point is strictly outside all other disks. Since $\mathcal{F}$ is dual to the Voronoi diagram, points $c_i$ and $c_{i+1}$ are connected in $\mathcal{F}$.

Let $T = \bigcup_{i=1}^n[c_i, c_{i+1}]$ be the drawing of the $\mathcal{P}$-tree on the plane. We also define $$E = T\cup\bigcup\{F_i\,\colon\, R(F_i) < 1\}.$$

\textbf{Remark.} The set of all Delaunay triangles and edges with \emph{covering radius} not exceeding $1$ (that is, triangles and edges that can be covered by a disk of unit radius) corresponds to a simplicial complex known as \emph{$\alpha$-complex}~\cite{alpha_shapes}. We, however, are interested in circumradius instead of covering radius. It is also crucial that the circumradius of a triangle must be strictly less than $1$.

\begin{lemma}\label{lemma:small-R-is-good}
Let $\triangle pqr$ be a triangle. For any point $t$, denote by $B_1(t)$ the open disk of unit radius centered at $t$. Let $K_q$ be the convex cone $\{q + \alpha(p - q) + \beta(r - q)\,\colon\,\alpha, \beta\geq 0\}$, and let $s_q$ be the arc $\partial B_1(q)\cap K_q$. If $R(\triangle pqr) < 1$, then $s_q\subset B_1(p)\cup B_1(r)$.
\end{lemma}

\begin{proof}
Let $C_q$ be the part of the Voronoi cell of points $\{p, q, r\}$ corresponding to the point $q$, restricted on $K_q$. Let $o$ be the circumcenter of $\triangle pqr$. It can be shown that $C_q$ lies inside the circle with diameter $oq$. Since $|o - q| = R(\triangle pqr) < 1$, it implies that this circle, in turn, belongs to $B_1(q)$. Therefore, $s_q$ is covered by Voronoi cells of $p$ and $r$; thus, for every point $t$ of $s_q$ either $|t - p|$ or $|t - r|$ is less than $|t - q|$, which is $1$.
\end{proof}

\begin{lemma}\label{lemma:E-is-connected}
$E$ is simply connected, or, equivalently, $\mathbb{R}^2\setminus E$ is connected.
\end{lemma}

\begin{proof}
Note that every triangle with circumradius less than $1$ has to be in a single region.
% Indeed, in every such triangle not intersecting $T$ one of the angles has to be at least $2\pi / 3$, and, since all its sides are at least $1$, its curcumradius is also at least $1$. Therefore, from now on we can only consider the triangles belonging to a single region.
Indeed, suppose that some Delaunay triangle $pqr$ is not contained in any $R_{ij}$. This means that for some $2$-disk $D_k$ the triangle $pqr$ intersects the segment $[c_k, f_k]$, where $f_k = c_k + \frac{c_k}{|c_k|}$.
Since the segment $[p, q]$ has to intersect the border between regions, and it cannot intersect $T$, we can assume, without loss of generality, that $[p, q]$ intersects $[c_k, f_k]$. We may additionally assume that neither $p$ nor $q$ coincides with $c_k$.

As shown in the proof of Observation~\ref{lemma:far}, angles $\angle pc_k0$ and $\angle 0c_kq$ do not exceed $2\pi/3$. Since $p$ and $q$ are at the different sides from line $\left<c_k\right>$, we have $\angle pc_kq\geq2\pi/3$. This also implies that $|p - q|\geq \sqrt{3}$.

If $r = c_k$, then $R(\triangle pc_kq)\geq 1$ due to the law of sines. Otherwise we can assume that $r$ and $c_k$ are at the different sides from the line through $p$ and $q$. Since $\triangle pqr$ is a Delaunay triangle, the circumcircle of $\triangle pqr$ does not contain $c_k$. Therefore, $\angle pqr\leq\pi/3$, and due to the law of sines, $2R(\triangle pqr)\geq \frac{\sqrt{3}}{\sqrt{3}/2} = 2$.

Therefore, it suffices to prove the lemma separately for each subsegment $\mathcal{D}_{ij}$. More specifically, we need to prove that among all Delaunay faces within the corresponding region no triangle with small circumradius ``blocks'' a triangle with large circumradius. The case when $j = i + 2$ is trivial, so we will stick to the case when $j = i + 4$. For simplicity, we may assume that $i = 1$ and $j = 5$.

We will keep in mind that $|c_k - c_{k+1}| = 1$ for every $k$. In particular, the inequality $R(\triangle c_{k-1}c_kc_{k+1}) < 1$ is equivalent to the inequality $\angle c_{k-1}c_kc_{k+1} < 2\pi/3$. We will also only consider triangles which are in $R_{ij}$.

Suppose that a triangle with circumradius smaller than $1$, merged with $T$, divides the plane into two components, of which the bounded one contains a Delaunay triangle with circumradius at least $1$. Without loss of generality, we can assume that one of the following cases takes place; see Figure~\ref{fig:delaunay-blocking-cases} (in all cases below, when we mention $R(\Delta)$, we assume that $\Delta$ is a Delaunay triangle).

\begin{figure}[h!]
    \centering
    \begin{subfigure}[t]{.48\textwidth}
    \includegraphics[width=.95\textwidth]{pics/delaunay-blocking-case-1.mps}
    \end{subfigure}
    \begin{subfigure}[t]{.48\textwidth}
    \includegraphics[width=.95\textwidth]{pics/delaunay-blocking-case-2.mps}
    \end{subfigure}
    \begin{subfigure}[t]{.48\textwidth}
    \includegraphics[width=.95\textwidth]{pics/delaunay-blocking-case-3.mps}
    \end{subfigure}
    \begin{subfigure}[t]{.48\textwidth}
    \includegraphics[width=.95\textwidth]{pics/delaunay-blocking-case-4.mps}
    \end{subfigure}
    \caption{Cases from Lemma~\ref{lemma:E-is-connected}. Gray triangles have circumradii less than $1$.}
    \label{fig:delaunay-blocking-cases}
\end{figure}

\begin{itemize}
    \item $R(\triangle c_{k-2}c_{k-1}c_k)\geq 1 > R(\triangle c_{k-2}c_kc_{k+1})$ for $k = 3$ or $k = 4$. % By the law of sines, $\sin \angle c_{k-2}c_{k-1}c_k < \sin \angle c_kc_{k+1}c_{k-2}$; hence, the quadrilateral $c_{k-2}c_{k-1}c_kc_{k+1}$ is convex.
    In this case the first inequality implies that $\angle c_{k-2}c_{k-1}c_k\geq 2\pi/3 > \pi/2$. Hence, the quadrilateral $c_{k-2}c_{k-1}c_kc_{k+1}$ is convex, and the inequality $\angle c_{k-2}c_{k-1}c_k\geq \pi/2$ contradicts Lemma~\ref{lemma:no-obtuse-in-delaunay}.
    
    \item $R(\triangle c_1c_4c_5) < 1$, $c_1c_2c_3c_4$ is a quadrilateral, which is divided by $\mathcal{F}$ into two triangles, both having circumradius at least $1$. In this case, since neither $\triangle c_1c_2c_3$ nor $\triangle c_2c_3c_4$ have circumradius less than $1$ due to Lemma~\ref{lemma:delaunay-sorted-process}, we conclude that both $\angle c_1c_2c_3$ and $\angle c_2c_3c_4$ are at least $2\pi/3$. Then $|c_1 - c_4|\geq 2$, which contradicts the fact that $R(\triangle c_1c_4c_5) < 1$.
    
    \item $R(\triangle c_1c_4c_5) < 1$, $R(\triangle c_1c_2c_3) < 1$, $R(\triangle c_1c_3c_4)\geq 1$. By the law of sines for $\triangle c_3c_1c_4$,
    $$\frac{|c_3 - c_4|}{\sin\angle c_3c_1c_4} \geq 2 \Rightarrow \sin\angle c_3c_1c_4\leq\frac12.$$
    Since the shortest side of $\triangle c_1c_3c_4$ is $c_3c_4$, this excludes the possibility that $\angle c_3c_1c_4\geq5\pi/6$, leaving us with $\angle c_3c_1c_4\leq\pi/6$.
    
    Then, as $R(\triangle c_1c_2c_3) < 1$, we know that $|c_1 - c_3| < \sqrt{3}$. By the law of sines for $\triangle c_1c_4c_3$,
    $$\frac{|c_1 - c_3|}{\sin\angle c_1c_4c_3} \geq 2 \Rightarrow \sin\angle c_1c_4c_3 < \frac{\sqrt 3}{2}.$$ Since $\angle c_1c_4c_3\leq \pi - \angle c_1c_2c_3\leq2\pi/3$, we have $\angle c_1c_4c_3 < \pi/3$. Finally, by Lemma~\ref{lemma:no-obtuse-in-delaunay}, $\angle c_1c_3c_4 < \pi/2$, which leads to $\triangle c_1c_3c_4$ having angles with sum less than $\pi$, thus a contradiction.
    
    \item $R(\triangle c_1c_4c_5) < 1$, $R(\triangle c_2c_3c_4) < 1$, $R(\triangle c_1c_2c_4)\geq 1$. This case is completely analogous to the previous one.
\end{itemize}
% Since in all cases we came to a contradiction, we establish the simple connectedness of $E$.
\end{proof}

According to Lemma~\ref{lemma:E-is-connected}, all triangles $\Delta$ of $\mathcal{F}$ with $R(\Delta) < 1$ can be ordered in such a way, that, if we add them to $T$ one by one, the union is always simply connected. Formally, if $\mathcal{F} = \{F_1, \ldots, F_m\}$, define $\mathcal{F}_k = \{F_1, \ldots, F_k\}$. Also define $E_0 = T$ and $E_k = E_{k-1}\cup F_k$ for all $k\in[m]$. In particular, $\mathcal{F}_0 = \varnothing$, $\mathcal{F}_m = \mathcal{F}$, and $E_m = E$. Then, due to Lemma~\ref{lemma:E-is-connected}, we may assume that all $E_i$ are simply connected. In particular, this means that for all $i\in[m]$ the triangle $F_i$ may be represented as $pqr$, where $p$, $q$, $r$ are three consecutive vertices of $\partial E_i$ in this order.

Let $(d_1, \ldots, d_k) = (c_{i_1}, \ldots, c_{i_k})$ be the cyclic sequence of all endpoints of the counterclockwise traversal of $\partial{E}$, and let $I = \{i_1, \ldots, i_k\}$.
% Define $pr(i_k) = i_{k-1}$ and $nx(i_k) = i_{k+1}$.
For all $j$ define $s_j$ as the arc of $B_{i_j}$ within the angle going counterclockwise from ray $[d_jd_{j-1})$ to ray $[d_jd_{j+1})$. If $d_{j-1} = d_{j+1}$ then we assume that $s_j$ has length $2\pi$, not $0$.
Developing the notation of Lemma~\ref{lemma:small-R-is-good}, for all $j$ define $K_j$ as the cone
$$d_j + \{\lambda x\,\colon\,\lambda\geq 0, x\in s_j\}.$$
In particular, $s_j = \partial B_{i_j}\cap K_j$. Note that, unlike in Lemma~\ref{lemma:small-R-is-good}, the cone $K_j$ does not have to be convex.

Lemma~\ref{lemma:small-R-is-good} says that if $F_i = \triangle pqr$, then $s_q\subset B_1(p)\cup B_1(r)$. We need the inverse of this.

\begin{lemma}\label{lemma:covered-arc-implies-small-R}
% Let $p$, $q$ and $r$ be consecutive members of $(c_{i_j})_{j=1}^k$. Then
$s_j\not\subset B_{i_{j-1}}\cup B_{i_{j+1}}$.
\end{lemma}

\begin{proof}
For simplicity, let $p = d_{j-1}$, $q = d_j$, $r = d_{j+1}$. Also, define for consistency $s_q = s_j$ and $K_q = K_j$.

First of all, if $|s_j|\geq\pi$, then both $B_{i_{j-1}}$ and $B_{i_{j+1}}$ intersect $s_j$ over an arc of length at most $\pi/3$, which implies the required relation. Otherwise, assume that $\angle qrp > \pi/2$. This implies that the side $pq$ is the largest in $\triangle pqr$, thus is longer than $1$. Therefore, there is the triangle $F_i\in\mathcal{F}$ such that $F_i = \triangle psq$ for some $s$.

Since $s$ and $r$ are at different sides from $pq$, we have $\angle psq + \angle qrp\leq\pi$, as otherwise the circumcircle of $\triangle psq$ would contain $s$, and hence, $\triangle psq$ would not be a Delaunay triangle. But then $psqr$ is a convex quadrilateral, and, according to Lemma~\ref{lemma:delaunay-sorted-process}, $\triangle psq$ has the smallest circumradius among all triangles with vertices from $\{p, s, q, r\}$. Hence, triangles $\triangle qrp$ and $\triangle psq$ are the Delaunay triangles in the triangulation of $\{p, q, r, s\}$, and the inequality $\angle qrp > \pi/2$ contradicts Lemma~\ref{lemma:no-obtuse-in-delaunay}. Therefore, the case $\angle qrp > \pi/2$ is impossible; similar to the case $\angle rpq > \pi/2$.

Since none of $\angle qrp$ and $\angle rpq$ is obtuse,
the proof of Lemma~\ref{lemma:small-R-is-good} works in the opposite way.
Indeed, let $o$ be the circumcenter of $\triangle pqr$. We know that $o$ is inside $K_q$; thus, the segment $[o, q]$ intersects $s_q$. If $t$ is the intersection point, then $|t - p|\geq |t - q| = 1$ and $|t - r|\geq |t - q| = 1$; thus, $t\in s_q\setminus(B_1(p)\cup B_1(r))$.
\end{proof}

The following two lemmas are the main lemmas of the whole proof. Lemma~\ref{lemma:delaunay-covers-neighbors} states that $I$ is the sequence of all involved disks, and hence, we only need those disks to build a curve containing $\partial S$. The last lemma establishes the fact that the curve built as formulated in Claim~\ref{lemma:can-construct-gamma-ij} indeed covers $\partial S$.

\begin{lemma}\label{lemma:delaunay-covers-neighbors}
If $\mathcal{D}_{ij}$ is a subsegment and $D_k\in\mathcal{D}_{ij}$, then $k\in I$ if and only if $B_k$ is involved in $\sigma_{ij}$. In particular, if $D_i$ is a $2$-disk, then $i\in I$.
\end{lemma}

\begin{proof}
Denote $F_i = \triangle pqr$, let $K_q = \{q + \alpha(p-q) + \beta(r-q)\,\colon\,\alpha, \beta\geq 0\}$, and let $S_q = B_1(q)\cap K_q$. We claim that $S_q\setminus\triangle{pqr}\subset B_1(p)\cup B_1(r)$. Essentially this means that whatever is covered by $B_1(q)$ within $K_q$, it either belongs to $\triangle pqr$ or is covered by $B_1(p)$ or $B_1(r)$. Then, if a disk $B_{i_k}$ is not involved, its arc $s_k$ must be covered by disks $B_{i_j}$.

If one of $\angle qrp$ and $\angle rpq$ is at least $\pi/2$, then $S_q\setminus\triangle pqr = \varnothing$. Otherwise, the circumcenter $o$ of $\triangle pqr$ belongs to $K_q$, and, as in Lemma~\ref{lemma:small-R-is-good}, the part of the Voronoi cell of $q$ inside $K_q$ belongs to $\triangle pqr$. Therefore, $S_q\setminus\triangle pqr$ is covered by the other two Voronoi cells, which means that each of its points is less than $1$ away from at least one of $p$ and $r$.

So if $B_k$ is involved in some $\sigma_{ij}$, then $k\in I$; therefore, this holds if $D_k$ is a $2$-disk. Indeed, if $k\notin I$, then at some point we added $\triangle pqr$ where $q$ corresponds to $c_k$. Due to Lemma~\ref{lemma:small-R-is-good}, the arc of $q$ between the two rays was covered by $B_1(p)$ and $B_1(r)$. On the other hand, due to what we showed above, the point of involvement should be on the arc of $q$ at the moment of adding $\triangle pqr$.

What remains to prove is that if $\mathcal{D}_{ij}$ is a subsegment, then for any
% $i < k < j$, if $k\in I$, then $B_k$ is involved in $\sigma_{ij}$.
$k$ such that $i < i_k < j$, the disk $B_{i_k}$ is involved in $\sigma_{ij}$. Again, for simplicity we assume that $i = 1$ and $j = 5$.

We remind that, due to Lemma~\ref{lemma:only_inner}, if a disk is not involved, then its arc is covered by other disks from the same region. We also remind that $s_k$ is the arc of $B_{i_k}$ within the angle going counterclockwise from ray $[d_kd_{k-1})$ to ray $[d_kd_{k+1})$. If $p\in s_k$, we will call the arc of $B_{i_k}$ withing the angle from ray $[d_kd_{k_1})$ to ray $[d_kp)$ a \emph{prefix} of $s_k$, and the arc from $[d_kp)$ to $[d_kd_{k+1})$ a \emph{suffix} of $s_k$.

We know that $B_{i_{k-1}}$ and $B_{i_{k+1}}$ intersect $s_k$ by some prefix and some suffix of this arc. Since $R(\triangle d_{k-1}d_kd_{k+1})\geq 1$ due to Lemma~\ref{lemma:delaunay-sorted-process}, we know that these two neighbors do not cover $s_k$ entirely due to Lemma~\ref{lemma:covered-arc-implies-small-R}. Therefore, if $B_{i_k}$ is not involved in $\sigma_{ij}$, then there must be another disk $B_l$ covering a point of $s_k\setminus(B_{i_{k-1}}\cup B_{i_{k+1}})$. Without loss of generality we may assume that $l < i_k$. Consider all possible cases, see Figure~\ref{fig:delaunay-covers-neighbors-cases}.

\begin{figure}[h!]
    \centering
    \begin{subfigure}[t]{.26\textwidth}
    \includegraphics[width=.95\textwidth]{pics/delaunay-covers-neighbors-case-1.mps}
    \caption{$l = i_k - 2$}
    \label{fig:delaunay-covers-neighbors-case-a}
    \end{subfigure}
    \begin{subfigure}[t]{.35\textwidth}
    \includegraphics[width=.95\textwidth]{pics/delaunay-covers-neighbors-case-2.mps}
    \caption{$i_{k-1} = i_k - 1$, $i_{k-2} = i_k - 3$}
    \label{fig:delaunay-covers-neighbors-case-b}
    \end{subfigure}
    \begin{subfigure}[t]{.35\textwidth}
    \includegraphics[width=.95\textwidth]{pics/delaunay-covers-neighbors-case-3.mps}
    \caption{$i_{k-1} = i_k - 2$, $i_{k-2} = i_k - 3$}
    \label{fig:delaunay-covers-neighbors-case-c}
    \end{subfigure}
    \caption{Cases from Lemma~\ref{lemma:delaunay-covers-neighbors}.}
    \label{fig:delaunay-covers-neighbors-cases}
\end{figure}

\begin{itemize}
    \item $i_{k-1} = i_k - 1$, $l = i_{k-2} = i_k - 2$. Since $B_{i_{k-1}}$ covers an arc of length $\pi/3$ of $s_k$, and $B_l$ covers the prefix of length $\pi - \angle{d_{k-2}d_{k-1}d_k}$, we have $\angle{d_{k-2}d_{k-1}d_k} < 2\pi/3$, which contradicts the fact that $R(\triangle d_{k-2}d_{k-1}d_k)\geq 1$; otherwise our process of adding triangles in ascending order would add this triangle as well. In particular, in this case $\pi - \angle{d_{k-2}d_{k-1}d_k} > 0$ must hold. See Figure~\ref{fig:delaunay-covers-neighbors-case-a}.
    \item $i_{k-1} = i_k - 1$, $l = i_{k-2} = i_k - 3$. In all remaining cases, including this, $l = i_k - 3$, which implies that $l = 1$ and $i_k = 4$. By the law of sines for $\triangle c_1c_4c_3$,
    $$\frac{|c_1 - c_3|}{\sin\angle c_1c_4c_3}\geq 2,$$
    hence,
    $$\sin\angle c_1c_4c_3 < \frac{\sqrt{3}}{2},$$
    therefore,
    $$\angle c_1c_4c_3 < \frac{\pi}{3}\text{ or }\angle c_1c_4c_3 > \frac{2\pi}{3}.$$
    % \begin{align*}
    %     \frac{|c_1 - c_3|}{\sin\angle c_1c_4c_3}\geq 2 & \Rightarrow \sin\angle c_1c_4c_3 < \frac{\sqrt{3}}{2} \\
    %     & \Rightarrow \angle c_1c_4c_3 < \frac{\pi}{3}\text{ or }\angle c_1c_4c_3 > \frac{2\pi}{3}.
    % \end{align*}
    If $\angle{c_1c_4c_3} > 2\pi/3$, then $|c_1 - c_3| > \sqrt{3}$, which contradicts the fact that $\triangle c_1c_2c_3\in\mathcal{F}$, which leaves the only option $\angle c_1c_4c_3 < \pi/3$.
    
    Let $t\in s_k$ be the point such that $B_3$ covers the arc of $s_k$ between $c_3$ and $t$. In particular, $\triangle c_3c_4t$ is regular. Using our assumption that $B_1$ covers something on $s_k$ except the arc between $c_3$ and $t$, and the fact that $\angle c_3c_4c_1 < \pi/3$, we conclude that $|c_1 - t| < 1$ must hold. But then some point of the segment $[t, (c_4 + c_3)/2]$ must be the circumcenter of $\triangle c_1c_3c_4$. Since it is not $t$, we have $R(\triangle c_1c_3c_4) < 1$, which leads to a contradiction. See Figure~\ref{fig:delaunay-covers-neighbors-case-b}.
    \item $i_{k-1} = i_k - 2$, $l = i_{k-2} = i_k - 3$. By the law of sines for $\triangle c_1c_4c_2$, we have
    $$\frac{|c_1 - c_2|}{\sin\angle c_1c_4c_2}\geq 2,$$
    hence,
    $$\sin\angle c_1c_4c_2 \leq \frac{1}{2},$$
    therefore,
    $$\angle c_1c_4c_2 \leq \frac{\pi}{6}\text{ or }\angle c_1c_4c_2 \geq \frac{5\pi}{6}.$$
    % \begin{align*}
    %     \frac{|c_1 - c_2|}{\sin\angle c_1c_4c_2}\geq 2 & \Rightarrow \sin\angle c_1c_4c_2 \leq \frac{1}{2} \\
    %     & \Rightarrow \angle c_1c_4c_2 \leq \frac{\pi}{6}\text{ or }\angle c_1c_4c_2 \geq \frac{5\pi}{6}.
    % \end{align*}
    Since $c_1c_2$ is the minimum side of $\triangle c_1c_2c_4$, we have the inequality $\angle c_1c_4c_2\leq\pi/3$; therefore, $\angle c_1c_4c_2\leq\pi/6$. Since $\triangle c_2c_3c_4$ was added, we know that
    $$\angle c_3c_4c_2 > \pi/6\Rightarrow \angle c_1c_4c_2 < \angle c_3c_4c_2.$$
    
    Let $t\in s_k$ be the point such that $B_2$ covers the arc of $s_k$ between $c_3$ and $t$. In particular, $c_2c_3c_4t$ is a rhombus. Using our assumption that $B_1$ covers something on $s_k$ minus the arc between $c_3$ and $t$, and the fact that $\angle c_2c_4c_1 < \angle c_2c_4c_3 = \angle c_2c_4t$, we conclude that $|c_1 - t| < 1$ must hold.
    
    Analogously to the previous case, $\angle{c_2c_1c_4} < \pi/3$. Therefore, $\angle c_2c_1c_4 < \pi/2$, which is equivalent to the inequality
    $$|c_1 - (c_2 + c_4)/2| > |(c_2 - c_4)/2|.$$
    But then, due to continuity, some point of the segment $[t, (c_2 + c_4)/2]$ must be the circumcenter of $\triangle c_1c_2c_4$. Since it is not $t$, we have $R(\triangle c_1c_2c_4) < 1$, which leads to a contradiction. See Figure~\ref{fig:delaunay-covers-neighbors-case-c}.
    \item $i_{k-1} = i_k - 1$, $l = i_{k-3} = i_k - 3$. Since $R(\triangle c_2c_3c_4)\geq 1$ and $R(\triangle c_1c_2c_3)\geq 1$, both $\angle c_2c_3c_4$ and $\angle c_1c_2c_3$ are at least $2\pi/3$. Therefore, $|c_1 - c_4|\geq 2$, which contradicts that $s_k\cap B_1\neq\varnothing$.
\end{itemize}

As we got a contradiction in each case, we conclude that $s_k$ is covered by $B_{i_{k-1}}$ and $B_{i_{k+1}}$.
\end{proof}

% To build a curve $\gamma_{ij}$ from Claim~\ref{lemma:can-construct-gamma-ij}, it suffices to show that for any $i_k$ the arc $s_k$ is not completely covered by $B_{i_{k-1}}$ and $B_{i_{k+1}}$.

\begin{lemma}
If $\mathcal{D}_{ij}$ is a subsegment, and $p\in \sigma_{ij}\cap\partial{B_{i_k}}$ for some $D_{i_k}\in\mathcal{D}_{ij}$, then $p\in s_k$ and $p$ is not covered by any of $B_{i_{k-1}}$ and $B_{i_{k+1}}$.
\end{lemma}

\begin{proof}
By construction, $p\notin E$ because every point from $E$ is at distance less than $1$ from one of the points. This fact and Lemma~\ref{lemma:E-is-connected} immediately imply that $p\in s_k$.

Similarly, since $|p - c_{i_{k-1}}|\geq 1$ and $|p - c_{i_{k+1}}|\geq 1$, we also have that $p$ is not covered by any of $B_{i_{k-1}}$ and $B_{i_{k+1}}$.
\end{proof}

This finishes the proof of Claim~\ref{lemma:can-construct-gamma-ij}.

% This means that the following algorithm builds a sparse-centered curve $\gamma_{ij}$ satisfying the required properties:

% \begin{enumerate}
%     \item Start at $f_i$,
%     \item Go counterclockwise along $B_{i}$ until the first intersection with $B_{nx(i)}$,
%     \item Go counterclockwise along $B_{nx(i)}$ until the first intersection with $B_{nx(nx(i))}$,
%     \item $\ldots$
%     \item Go counterclockwise along $B_{nx(\cdots(nx(i))\cdots)} = B_j$ until $f_j$.
% \end{enumerate}

\end{appendices}
